\specialchapt{ABSTRACT}

A software defined radio is defined as a communication system where components of a communication system that are typically done in hardware are now done in software.  The result is a highly flexible communication system that can adapt to changes to the system based on requests by the user or due to conditions in the radio frequency channel.  Software Defined Radios (SDRs) have been used for a variety of applications, mostly in the area of communications.  However, they have not been widely applied to remote sensing applications such as a radiometer.  A SDR based radiometer offers a very flexible and robust system more so than a more traditional radiometer which has fixed hardware and usually little room for flexibility and adaptability based on the needs of the radiometer application or due to changes in the radiometers environment.  The price of SDRs have also been steadily dropping making implementing them into radiometers a more attractive option compared to using traditional RF hardware.  In this thesis we will look into the feasibility and theory of using an off the shelf SDR hardware platform for a radiometer application.  In addition, we will look into how we can use the GNURadio software to create a total power radiometer within software and how this software can be used to make easy changes to the functionality of the radiometer.  Finally we will look at the preliminary results obtained from laboratory tests of the SDR radiometer system.
