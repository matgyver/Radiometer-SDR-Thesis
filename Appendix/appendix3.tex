% An example second appendix from the example thesis thesis.tex.
\chapter{Direct-Sampling Digital Correlation Radiometer}

\section*{Theory of Operation}

The original ISU Radiometer was constructed by the University of Michigan and was based on Dr. Mark Fischman's dissertation on a Direct-Sampling Digital Correlation Radiometer [\cite{Fischman2001}].  This type of radiometer amplifies the signal but then immediately digitizes the signal to be processed.  In many ways, this is how a software defined radio works, however in this case the ISU radiometer is under-sampled.  However, as explained by Dr. Fischman this is acceptable for a total power radiometer since the only information we need is power information.  The radiometer that Dr. Fischman proposes also correlates the signals, one from the vertical polarization and the other from the horizontal polarization from the antenna.  

\subsection{Implantation in the ISU Radiometer}
The original ISU radiometer worked by under-sampling the RF signal coming into the analog to digital converters.  This under-sampled signal lacks enough sample to recreate the full signal, however the power information can be extracted.  The original ISU radiometer also used both the vertical and the horizontal polarization coming from the antenna. This allowed the radiometer to correlate between the vertical and horizontal polarization.  
