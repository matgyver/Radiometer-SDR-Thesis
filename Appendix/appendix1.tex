% Appendix1 file from standard thesis template
\appendixtitle
\appendix
\chapter{Source code}

The following is the Python code used with the Ettus N200 software defined radio.  This python code should work on any platform that has the GNURadio libraries installed and also the USRP drivers for communicating with the N200.  This code can also be easily modified to communicate with any SDR that GNURadio can communicate with.  

This script was developed using GNURadio Companion.  This copy of the script may be out of date.  The latest version of the script file can be found on the author's personal website, \url{http://www.rfgeeks.com/research/master-s-thesis}.


\section*{Python code for total power radiometer}
\lstset{
  language=Python,
  showstringspaces=false,
  formfeed=\newpage,
  tabsize=4,
  commentstyle=\itshape,
  basicstyle=\ttfamily,
  morekeywords={models, lambda, forms}
}

\newcommand{\code}[2]{
  \hrulefill
  \subsection*{#1}
  \lstinputlisting{#2}
  \vspace{2em}
}

\code{Total Power Radiometer}{Code/N200_TPR.py}

\section*{Matlab code for reading and displaying data from GNURadio}
\lstset{
  language=Matlab,
  showstringspaces=false,
  formfeed=\newpage,
  tabsize=4,
  commentstyle=\itshape,
  basicstyle=\ttfamily,
  morekeywords={models, lambda, forms}
}

\newcommand{\matlabcode}[2]{
  \hrulefill
  \subsection*{#1}
  \lstinputlisting{#2}
  \vspace{2em}
}

\newpage

\matlabcode{GNURadio Parsing Code}{Code/gnuradio_parse.m}
