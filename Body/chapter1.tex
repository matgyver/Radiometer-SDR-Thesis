% Chapter 1 of the Thesis Template File
\chapter{INTRODUCTION AND  BACKGROUND}

%This is the opening paragraph to my thesis which explains in general terms the concepts and hypothesis which will be used in my thesis.

%This section will cover the basic and general concepts of my thesis.  This will be a high level approach to the problem and to the proposed solution.  In addition, this section gives a general background to the project

%This chapter has 4 sub-sections.  1)  Introduction to SDRs, 2) Introduction to GNURadio and 3) Introduction to the current ISU radiometer and finally the 4th is the Related Works section

\section{Introduction}
The goal of this thesis is to explore the use of a software defined radio that can function as well as or better than current radiometers.  In addition, we aimed to develop a radiometer that is more flexible than most radiometers and still maintain the accuracy and stability of a traditional radiometers if not exceed these specifications.  A secondary goal was to use off the shelf components that are generally more accessible and often less expensive.  This allows radiometers to be more accessible to a wider scope of researchers in this field.  And finally a tertiary goal was to ensure that the system as a whole is easy to use.  This ties to our secondary goal of making radiometers more accessible to a wider range of researchers and research topics.

This thesis looks to explore the following questions: (1) Can we use an off the shelf SDR along with GNURadio to recreate a radiometer in software that is easy to use and cost effective?  (2) If so, what performance can we get from the system?  (3) What benefits do we gain (if any) from using a SDR from a more traditional radiometer? The results of this research and experimentation are the subject of this thesis.

\subsection{Motivation}
Remote sensing allows us to collect information about the world that is around us.  This information can give us valuable information such as vegetation health, soil moisture, ocean salinity, and astronomy.  Using radiometers for remote sensing is fairly new, starting around 1960[\cite{Ulaby}].  

Using radiometers for remote sensing has several advantages as we are able to look beyond certain obstacles that would otherwise be blocked in other remote sensing methods such as photography.  Radiometers have been used in the past for primarily astronomy in scanning for celestial objects that may be hidden from visual telescopes.  More recently radiometers have now been focusing on Earth and helping scientists to better understand the water cycle on Earth by monitoring our ocean salinity and soil moisture.  

\subsection{Problem Statement}
Radiometers have proven to be an excellent tool in remote sensing and have been used in a variety of research applications.  While radiometers have been around for quite some time, the equipment and method of implementing a radiometer has not changed much over the last fifty years.  However, radiometers have not seen much widespread use.  This largely due to the fact that there are very few commercially built radiometers, most radiometers are built to fit a specific purpose or research area.  This is in part due to that different tasks may require changes to the radiometer hardware.  This results in radiometers that are often expensive and require a certain amount of upkeep.  

If though, a radiometer can be built with commercial off the shelf components and also be flexible to adapt to different research or experimental needs, then that type of radiometer will be more desirable and will reduce costs by using available COTS solutions.  

The solution to this problem, as outlined in this thesis, is to create a software defined radiometer that digitizes the incoming radio frequency information as soon as possible and then process this information in software instead of using hardware.  In addition, we will use existing hardware and tools that are commercial available and will be lower in cost or free to the user.  

\subsection{Contributions}
Contributions from the work performed for this thesis include a development of a software defined radiometer that is capable of performing as a radiometer and meets or exceeds the performance of a traditional radiometer.  This includes source code written in the open source software defined radio framework known as GNURadio.  Additional contributions include analysis tools used to analyze data obtained from the software defined radiometer.  Finally contributions were made in using a different technique to mitigate interference that impacts the radiometer readings.  This is accomplished differently from other methods because unlike other radiometers we have both amplitude and frequency information to identify and mitigate this interference.

\subsection{Thesis Organization}
This thesis is organized to first introduce the user to the background and short introduction to both radiometers and to software defined radios.  Next, a more detailed explanation of the theory that allows for a software defined radio to emulate a radiometer in software followed by details on how key components of a radiometer are implemented in software.  We then examine experimental data that was collected to confirm our hypothsis of software defined radiometer and explore advantages a software defined radiometer has over a more traditional radiometer.  Finally we will analyze our results obtained from our experimental data and ensure it meets our expectations for a software defined radiometer.

\subsection{Software Defined Radio Radiometer}

A Software defined radio consist of both hardware and software that allow it to perform the operations of a radio or communication channel.  A software defined radio used for radiometer applications is identical to a software defined radio used for, as an example, a 802.11b radio with one major difference.  Since we need to amplify the signal more than what most communication applications require; we do require more powerful or additional LNAs to boost this signal.  In addition, since the first LNA plays a major role in the overall system noise and this system noise does affect performance of the radiometer, the selection of this LNA is important.  However, all other components are the same components used in other applications.

A software defined radio radiometer behaves analogous to a more traditional radiometer and thus the application is the same as a traditional radiometer.  This includes applications such as radio astronomy that includes applications in Earth Science such as soil moisture and ocean salinity[\cite{Ruf}].  A software defined radio radiometer can also allow for new application development that can expand the remote sensing field.  Since we have moved the majority of the hardware to software this allows us to further shrink the size and weight of the radiometer.  This allows for other radiometer applications such as Unmanned Aerial Vehicles (UAVs) for scanning soil moisture and ocean salinity remotely[\cite{McIntyre}].  

We will now introduce the three major components that make up a software defined radio radiometer.  

%\textbf{N200 SDR}
\subsubsection{N200 Software Defined Radio} 
The key component for a software defined radio radiometer is the software defined radio or SDR that will do most of the work as a radiometer through software.  The equipment selected for researching into this topic was the Ettus Research Group N200 SDR.  The SDR selected utilizes daughter boards as the RF front end to the SDR and up to two daughter boards may be installed into a N200.  This is an important consideration as one of the requirements is to be able to look at both the V-Pol and the H-Pol signal coming from the antenna.  By having a dual receiver SDR it is possible to correlate the signal and other signal analysis can also be done.  Another important reason the N200 is selected was due to its ability to handle up to 50 MHz of bandwidth to the computer and up to 25 MHz of RF bandwidth per daughter-board plugged in to the SDR.  This means that it is possible to have two receive cards that can stream up to 25 MHz bandwidth each.  

The N200 utilizes a flexible architecture for a variety of RF interface systems based on the frequency range desired and if receive and/or transmission is needed.  These daughter boards directly receive the RF signal and then outputs the analog I and Q signals that are then sampled by the N200 A/D converter for reception or receives the I and Q values from the N200 D/A converter for transmission. 

{\begin{figure}[h!tb] 
\centering
\includegraphics[width=14cm]{Images/n200_block_edited}
\isucaption{A block diagram of the Ettus N200 SDR}
\label{N200_block}
\end{figure}
}

The daughter board selected is the DBSRX2 card as this card is receive only and operates between 800 MHz and 2.4 GHz.  The DBSRX2 also has built in amplification that is adjustable through software.

\subsubsection{GNURadio Software}

For the software portion of the radio, we settled to use GNURadio, an open source software package that is well supported by the community and by the Ettus Research Group and the N200 SDR.  GNURadio also comes with what is known as GNURadio Companion or GRC.  This program provides us with a GUI interface and allows for the drag and drop of blocks that represent certain functions that can be used with the SDR.  GNURadio and GRC use Python as its main scripting language and GNURadio uses C++ code for directly accessing the hardware.  The hardware interface for the N200 is provided by Ettus through drivers that allow GNURadio to talk to the hardware.  Like GNURadio, these drivers are also available to all platforms.  Ettus has released these drivers to the open source community and continues to support the hardware drivers and GNURadio integration.

GNURadio operates on multiple platforms including Windows, Mac OS X, and Linux.  Linux is by far the most popular platform to work on and most of this thesis research was completed within the Linux environment.  Testing is also done though with a MacBook Pro running OS X 10.9.  The OS X implementation is well supported and is installed through MacPorts.  While windows is technically supported, it is often difficult to install and many of the libraries required are not 64-bit.  For these reasons windows was not used for executing GNURadio, but was used for some of the data analysis using Python.

Through GNURadio, we can now write code that will take the data given to us from the SDR and manipulate the signal as we need to mimic a radiometer.  The power detection, filtering and recording of this data is all done through GNURadio.  This also means that we are shifting more of the computational power done on the signal from the FPGA to a computer running GNURadio.  There are ways to change this behavior and upload code directly to the FPGA.  However, for this thesis it was easier to debug and work with GNURadio by keeping the processing on the desktop computer.  It does however mean that the host computer must be powerful enough to handle the signal and specifically the large bandwidth that we wish to send to it.  

\subsection{RF Front End}
The RF front end plays a critical role in the radiometer as the LNAs used in the front end has a large impact on the system noise generated by the radiometer itself.  A traditional radiometer utilizes both amplification through the LNAs and also includes filtering to the desired bandwidth.  A SDR radiometer does not require the filters as we are able to create these in software, however the amplification stages need to remain.  

{\begin{figure}[h!tb] 
\centering
\includegraphics[width=0.8\linewidth]{Images/RF_Front_end.png}
\isucaption{The ADISIMRF program used to verify the design of the RF Front End}
\label{ISU_Rad}
\end{figure}
}
A typical RF front end uses a 3 stage Low Noise Amplifier (LNA) to amplifier the noise while keeping the noise contributed to the system as low as possible.  As with any radiometer, the first LNA is the most critical as it contributes the most to the overall system noise temperature.  For this reason a LNA that did not have a large gain but had a low noise figure is chosen. The second and third LNA has higher gain values at the cost of a higher noise figure, although not by much.  However, since they are further down the chain, they do not contribute as much to the total system noise.  The reason for this is further explained in chapter 3. 

\section{Background}
\textbf{Software Defined Radios} 
Software Defined Radios (SDRs) are used for a variety of applications but their primary application has been in the area of communications.  They appeal to applications where being able to change a modulation scheme or filter on the fly is desirable.  In these areas, SDRs often outperform a traditional hardware only radio with their ability to rapidly change their operations by simply changing their software.  Early SDRs were expensive due to the high costs in the analog to digital converters (ADCs) needed and in the high speed Field Programmable Gate Arrays (FPGA) used.  In recent years however, the cost of SDRs have decreased due to the cost of these key components decreasing in cost as well.  Even though the cost has gone down the performance of SDRs have increased and this has lead to new developments in applications for using SDRs in new and different ways.

The basic concept behind a SDR is that it will digitize the RF signal as soon as possible.  Once digitized, it is now evaluated by a computer, FPGA, or a dedicated System on Chip (SoC).  A canonical software defined radio architecture is one that consists of a power supply, antenna, multi-band RF converter, and a single chip that contains the needed processing and memory to carry out the radio functions in software [\cite{Mitola1995}]. This allows us to extract certain hardware functions, such as filtering, into the digital domain which can then be manipulated by software.  Since software is now manipulating the signal, we can rapidly change what functions we execute on the signal by changing the software.  This gives SDRs a high amount of flexibility as components that are normally done in hardware can now be done in software and can be changed by simply uploading new software or firmware to the system.  This also gives us a benefit in cost as certain components are no longer needed and changes done in software do not require additional hardware to be added or to be swapped out.

\textbf{Radiometers}  
Radiometers are radio receivers that simply listen to and record the amount of power received.  However, the power received is not a coherent signal, instead it is the amount of noise the radiometer sees.  Radiometers, at the basic level, listen to noise that is generated naturally from a source.  These sources can vary and the applications vary as well.  Some examples of radiometer applications have been in evaluating soil moisture content, ocean salinity levels, and celestial objects[\cite{ulaby2014}].

The amount of noise that is generated is due to the thermal agitation of the charge carriers, usually the electrons, which is directly correlated to the physical temperature of the source[\cite{Nyquist1928thermal}].  This correlation is done as a noise temperature.  All objects emit this noise and the intensity will vary on multiple parameters and on what the source is.  One source that has current research at Iowa State University is in detecting soil moisture.  Numerous soil types can be observed including sandy types of soil[\cite{Liu}], The brighter or warmer the noise temperature is, the more RF noise that has been received which correlates to a drier soil.  The less RF noise power received the cooler the noise temperature and this indicates wetter soil area. Radiometers such as these are already in service on satellites such as the Soil Moisture and Ocean Salinity (SMOS) satellite launched by the European Space Agency (ESA) and are used by scientists to monitor the Earth's soil moisture and ocean salinity[[\cite{McMullan}][\cite{Hardy}].  

A traditional radiometer uses multiple Low Noise Amplifiers (LNAs) to amplify the signal and filters to limit the bandwidth to a certain range.  Power is often measured using a square law detector and this is often an analog voltage output.  Even though we are measuring noise we want to measure the right noise and we want the noise generated by the radiometer hardware itself to be as low as possible.  In other words we are only interested in listening to the noise being generated from an outside source.  Additional noise in the system is impossible to eliminate, however we can take steps to reduce it as much as possible.  In addition, we can calculate what this noise is and take steps to account for it in our measurements.  However, this means that stability is another parameter within the radiometer.  If the noise generated within the radiometer is constantly changing then this makes it difficult to account for this additional noise.  There are steps we can take to work with this though.  A traditional method often used is the Dicke radiometer which switches between the measurement of the antenna and a known source[\cite{Dicke}].  By referencing this known source the Dicke radiometer can calibrate and account for any drift due to variations in the system.  Another method is to use highly stable components and keep them stable during the operation of the radiometer.  For LNAs, temperature directly effects the overall gain from the LNA.  In some radiometer applications we can control the temperature which allows us to keep the LNAs stable during the operation.  Stability will be discussed in greater detail later in this thesis.

Additional improvements to radiometers have also been done by digitizing parts of the radiometer.  The most common method for doing this is by digitizing the analog output of the square-law detector and sending that to a computer or processing unit to analyze the data[\cite{Bremer}].  While this does allow for easier computation and storage of the information it does not alleviate the needs to maintain stability or reduce possible additional noise of the system since this data is digitized after the RF signal chain.

\textbf{A more modern Radiometer}  
Iowa State University owns a 1.4 GHz, dual polarization, correlating radiometer.  This radiometer is in use by Dr. Brian Hornbuckle and his research team.  However, in recent years the radiometers digital circuitry has suffered from numerous issues which has made it unusable.  Part of the driving force behind this research in this thesis is to determine new methods that can be used to rebuild this radiometer.

The ISU Radiometer was built at the University of Michigan and put into service at ISU in 2006.  The ISU radiometer is unique in that it is one of the few direct sampling radiometers in use\cite{Erbas}.  This radiometer takes the RF signal, amplifies and filters the signal, and then sends it directly to an analog to digital converter.  At the time the ISU radiometer was built an A/D that is able to sample accurately at 1.4 GHz was expensive and hard to come.  However, the ISU radiometer does not sample at 2.8 GHz or above, instead it samples at 1.4 GHz.  The ISU radiometer is able to do this because it is only interested in power information.  This means that we are not interested in recreating the entire signal and therefore we can under-sample the signal.  The A/D data is then sent to a Field Programmable Gate Array (FPGA) which then processes the data.  The ISU radiometer is also a correlating radiometer which means that it looks at both the vertical polarization (V-Pol) and the horizontal polarization (H-Pol) and then correlates this information[\cite{Fischman2001}].  

\section{Related Works}
As mentioned before, software defined radios have been used in a number of applications.  While as far as the author has determined, this is the first application of using an off the shelf software defined radio for a radiometer in remote sensing for soil moisture measurements, there has been similar applications done.  The closest application that has been found is with radio astronomy.  Radiometers used in radio astronomy is nothing new and has been used for some time now[\cite{Ohm}]. There are similarities between radio astronomy and remote sensing of the ground.  Both are using a radiometer to listen to a source of interest.  With radio astronomy the basic principle is that a hot source such as a star will produce more noise than the cooler background of space.  In remote sensing we are looking at the overall change of the source to determine its characteristics.  Both cases are measuring the total power of the noise and based on that information we can determine some properties of the source we are looking at.

\subsection{Radio Astronomy Examples}
The Shirleys Bay Radio Astronomy Consortium (SBRAC) located in Smiths Falls, Ontario is using the USRP software defined radio in conjunction with GNURadio.  SBRAC has successfully used this configuration to obtain radio astronomy data by looking at the hydrogen line at 1420.4058 MHz [\cite{Leech2007}].  The person in charge of this facility, Marcus Leech, contributed software to the GNURadio specifically for radio astronomy applications.  It was this software branch that was used as the base for the GNURadio program that was used in this thesis.  Marcus Leech continue to contribute to the GNURadio community and continues to provide support for these functions as well[\cite{Leech}].

Another example of a software defined radio used as a radiometer is from students at the University of Illinois and Grand Valley State University that built a software defined radio to listen to emissions from Jupiter[\cite{Behnke}].  This software defined radio was built using an Analog Devices AD9460 and a Xilinx Spartan-3E-500 FPGA to build the SDR itself.  A RF front end was also built to filter and amplify the signal coming into the SDR.  Finally, this group also used GNURadio to interface to and talk to the SDR and used both Python and wxGUI to build a working interface.  The students reported that the SDR radiometer worked well and was able to do so at a low price point.

It should be noted that much of the related works found worked with using a radiometer for astronomy or for looking at the sky.  For the ISU radiometer however we are looking at the ground and we are using the radiometer for soil moisture instead of measuring stars and other points of interest in the sky.  While the fundamentals is the same for either radiometer some adjustments need to be made because a radiometer that is looking at the sky often sees a cool brightness temperature whereas a radiometer looking at the ground sees a much warmer brightness temperature.  This is due to the albedo of the Earth having a much warmer noise temperature then what you find with radio astronomy[\cite{Tiuri}].
\subsection{Other Digital Radiometers}

A digital radiometer is not a new concept.  Early radiometers often digitize the analog voltage information from the square-law detector and then send that information to a computer for storage or analysis.  

A pre-cursor to a software defined radio, some radiometers also digitize the incoming RF signal, but under-sample this information.  Since only power is the only information desired for these radiometers, this was acceptable.  However, these radiometers did often use the same components you might find in a software defined radio such as an A/D converter and FPGA.  These components however were used in different ways.

One reason why these devices were used differently from a SDR was due to cost.  Most radiometer operations happen at 1.4 GHz or above.  A/D converters at these higher frequencies become more expensive and harder to obtain.  In recent years however, these costs have come down.

The major difference between other digital radiometers and what is discussed in this thesis is that we retain both phase and magnitude information and instead mimics a traditional radiometer in software by summing and squaring the I and Q values and then running this information through a low-pass IIR filter.  By retaining this information, we can perform a more in-dept analysis of the signal coming into the radiometer which allows for greater agility in the system.

\subsection{RFI Mitigation}
Radio Frequency Interference (RFI) is a common problem with a nearly all radiometers.  This problem is often exasperated with satellite radiometers since those radiometers see a large area on the ground.  Previous missions such as the Soil Moisture Ocean Salinity (SMOS) satellite has numerous issues with RFI that skews the data that they collect.  There have been a number of methods to either work around the RFI or attempt to filter it using mechanical filters.  University of Michigan has done a number of experiments in detection and mitigation of RFI with the use of mechanical filters [\cite{DeRooRFI}].  And a number of papers have been published on the detection of RFI in radiometers [\citep{DeRoo}][\cite{Forte}].

In all of these cases the radiometers used do not retain frequency information and thus must use other methods to detect and try to mitigate the signal.  In this thesis we cover a way to mitigate RFI and are able to do so using software to define a filter to notch out the offending signal.  Detection in a software defined radio radiometer is easier as we have both power and frequency information to design the filter.