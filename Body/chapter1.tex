% Chapter 1 of the Thesis Template File
\chapter{INTRODUCTION AND  BACKGROUND}

%This is the opening paragraph to my thesis which explains in general terms the concepts and hypothesis which will be used in my thesis.

%This section will cover the basic and general concepts of my thesis.  This will be a high level approach to the problem and to the proposed solution.  In addition, this section gives a general background to the project

%This chapter has 4 sub-sections.  1)  Introduction to SDRs, 2) Introduction to GNURadio and 3) Introduction to the current ISU radiometer and finally the 4th is the Related Works section
\section{Background}
\textbf{Software Defined Radios.} Software Defined Radios (SDRs) have been used for a variety of applications, but their primary application has been in the area of communications.  They appeal to applications where being able to change a modulation/de-modulation scheme or filter on the fly is desirable.  In these areas, SDRs often outperform a traditional hardware only radio with their ability to rapidly change their operations by simply changing their software.  Early SDRs were expensive due to high costs in the A/Ds needed and in the Field Programmable Gate Arrays (FPGA) used.  In recent years however, the cost of SDRs have decreased due to the cost of these key components decreasing in cost as well.  Even though the cost has gone down, the performance of many SDRs have increased.  This has lead to new applications for SDRs being developed and using SDRs in new and different ways.

The basic concept behind a SDR is that it will digitize the RF signal as soon as possible.  Once digitized, it can now be evaluated by a computer, FPGA, or a dedicated System on Chip (SoC).  A canonical software radio architecture is one that consists of a power supply, antenna, multi-band RF converter, and a single chip that contains the needed processing and memory to carry out the radio functions in software [\cite{Mitola1995}].  This allows us to extract certain hardware functions, such as filtering, into the digital domain which can then be manipulated by software.  Since software is now manipulating the signal, we can rapidly change what functions we execute on the signal by changing the software.  This gives SDRs a high amount of flexibility as various components that are normally done in hardware can now be done in software and can be changed by simply uploading new software or firmware to the system.  

\textbf{Radiometers.}  Radiometers are radio receivers that simply listen to and record the amount of power received.  However, the power received is not a coherent signal, instead it is the amount of noise the radiometer sees.  Radiometers are passive devices, in other words no source signal is sent.  Radiometers at the basic level, listen to noise that is generated naturally from a source.  There are several uses for radiometers.  Radio astronomy and soil moisture measurements are two examples of radiometers used in remote sensing applications.  However, at Iowa State University there is active research in using radiometers for soil moisture measurements.  The amount of power we receive from the soil correlates to how much moisture is in the ground.  Therefore we will often reference this current work in this thesis.   

The amount of noise that is generated is due to the thermal agitation of the charge carriers, usually the electrons, which is directly correlated to the physical temperature of the source[\cite{Nyquist1928thermal}].  This correlation is done as a noise temperature.  The brighter or warmer the noise temperature is, the more RF noise that has been received which correlates to a drier soil.  The less RF noise power received, the cooler the noise temperature and this indicates wetter soil area. Radiometers such as these are already in service on satellites such as the Soil Moisture and Ocean Salinity (SMOS) satellite launched by the European Space Agency (ESA) and are used by scientists to monitor the Earth's soil moisture and ocean salinity[\cite{Liu}][\cite{McMullan}][\cite{Ruf}][\cite{McIntyre}][\cite{Hardy}].  

A traditional radiometer uses several Low Noise Amplifiers (LNAs) and filters and the power is often measured using a square law detector.  Even though we are measuring noise, we want to measure the right noise and we want the noise generated by the radiometer hardware itself to be as low as possible.  In other words, we are really listening to the noise being generated from an outside source.  Additional noise in the system is impossible to eliminate, however we can take steps to reduce it as much as possible.  We can also calculate what this noise is and take steps to account for it in our measurements.  However, this means that stability is another factor within the radiometer.  If the noise generated within the radiometer is constantly changing, this makes it difficult to account for this additional noise.  There are steps we can take to work with this though.  A traditional method often used is the Dicke radiometer which switches between the measurement of the antenna and a known source\cite{Dicke}.  By referencing this known source the Dicke radiometer can calibrate and account for any drift due to variations in the system.  Another method is to use highly stable components and keep them stable during the operation of the radiometer.  For LNAs, temperature directly effects the overall gain from the LNA.  In some radiometer applications we can control the temperature which allows us to keep the LNAs stable during the operation.  It is this approach that the current ISU radiometer uses by using thermal electric coolers (TEC) that  regulate the temperature of the LNAs to keep them stable.

Additional improvements to radiometers have also been done by digitizing parts of the radiometer.  The most common method for doing this is by digitizing the analog output of the square-law detector and sending that to a computer or processing unit to analyze the data[\cite{Bremer}].  While this does allow for easier computation and storage of the information, it does not alleviate the needs to maintain stability of the system since this data is digitized after the RF signal chain.

\textbf{ISU Radiometer.}  Iowa State University currently owns a 1.4 GHz, dual polarization, correlating radiometer.  This radiometer is currently in use by Dr. Brian Hornbuckle and his research team.  However, in recent years the ISU radiometer has not performed as expected and it was determined that the radiometer should be rebuilt using newer technologies.  

The ISU Radiometer was built at the University of Michigan and put into service at ISU in 2006.  The ISU radiometer is unique in that it is one of the few direct sampling radiometer in use\cite{Erbas}.  This radiometer takes the RF signal, amplifies and filters the signal, and then sends it directly to an analog to digital converter.  At the time the ISU radiometer was built, an A/D that could sample accurately at 1.4 GHz were expensive and hard to come by due to the fact that Nyquist's theorem states that we must sample at least two times the frequency.  However, the ISU radiometer does not sample at 2.8 GHz or above, instead it samples at 1.4 GHz.  The reason the ISU radiometer can do this is due to the fact that a radiometer is generally only interested in the power of the incoming signal.  This means that we are not interested in recreating the entire signal and therefore we can under-sample the signal.  The A/D data is then sent to a Field Programmable Gate Array (FPGA) which then processes the data.  The ISU radiometer is also a correlating radiometer, which means that it looks at both the V-polarization and the H-polarization and then correlates this information[\cite{Fischman2001}]. 

For the work conducted in this thesis, the ISU radiometer will be referred to as part of a traditional radiometer and is simply used to provide the amplified signal in the 1.4 GHz signal of interest.  Both the Analog to Digital converters and the FPGA have been disabled for all experiments done within this thesis.  This is due to the fact that these components were not functioning properly while work was conducted with this research.

\section{Introduction}
The goal of this thesis is to explore other methods that could be used for a remote sensing radiometer.  The end goal is to develop a radiometer that is more flexible than most radiometers and still maintain the accuracy and stability of a traditional radiometers if not exceed these specifications.  This includes ensuring that the SDR radiometer meets or exceeds what a typical radiometer is able to do.  This thesis also explores that value added to a radiometer by using a software defined radio.  A secondary goal was to use Commercial Off The Shelf (COTS) components as these components generally more accessible and are often lower cost than a custom solution.  This would allow radiometers to be more accessible to a wider scope of researchers in this field.  And finally a tertiary goal was to ensure that the system as a whole is fairly easy to use.  This ties to our secondary goal of making radiometers more accessible to a wider range of researchers and research topics.

%As stated in the background, ISU currently owns a radiometer that was used for the remote sensing of soil moisture.  In many ways, the current ISU radiometer is like a SDR.  The RF signal is digitized and then processed by a FPGA to give the total power reading needed for radiometry.  However, due to the under-sampling of the signal some information is lost.  The ISU radiometer assumes that incoming signal is free of any interference and the power recorded is from the target of interest.  However, in recent years the ISU radiometer has not been performing as expected, and without additional information it is almost impossible to diagnose these issues out in the field.  The ISU radiometer is also not frequency agile, it is fixed at 1.4 GHz and the bandwidth is also fixed at 20 MHz.  All of this means that if an interfering signal is present the current radiometer would not be able to determine this and even if it could, it has no means to try to avoid the signal.  

This thesis looks to explore the following questions: (1) Can we use a SDR along with GNURadio to recreate a radiometer in software?  (2) If so, what performance can we get from the system?  (3) What benefits do we gain (if any) from using a SDR from a more traditional radiometer?  The results of this research and experimentation are the subject of this thesis.


\subsection{Software Defined Radios}

Although a Software Defined Radio uses software to do the bulk of the signal processing, some hardware is still needed.  Specifically hardware is needed to capture the signal and digitize it.  A FPGA is often used to process the signal before passing it to a computer for further processing however in some cases a FPGA or a System on a Chip may do all the processing of the signal.

\textbf{N200 SDR} The equipment that was selected for researching into this topic was the Ettus Research Group N200 SDR.  This SDR uses daughter boards as the RF front end to the SDR and up to two daughter boards may be installed into a N200.  This is an important consideration as one of the requirements is to be able to look at both the V-polarization and the H-polarization signal coming from the antenna.  This allows us to correlate the signal and other signal analysis can also be done.  Another important reason the N200 was selected was due to the fact that it can handle up to 50 MHz of bandwidth to the computer.  This means that it is very possible to have two receive cards that can stream up to 25 MHz bandwidth each.  

The N200 utilizes a flexible architecture for a variety of RF interface systems based on the frequency range desired and if receive and/or transmission is needed.  These daughter boards directly receive the RF signal and then outputs the analog I and Q signals that are then sampled by the N200 A/D converter for reception or receives the I and Q values from the N200 D/A converter for transmission. 

 {\begin{figure}[h!tb] 
\centering
\includegraphics[width=14cm]{Images/n200_block_edited}
\isucaption{Block diagram of the N200 SDR}
\label{N200_block}
\end{figure}
}

The daughter board selected was the DBSRX2 card.  This card is a receive only card that operates between 800 MHz and 2.4 GHz and would thus work in the 1.4 GHz band we are interested in.  The DBSRX2 also has built in amplification that is adjustable through software.

\subsection{GNURadio Software}

For the software, we settled to use GNURadio, an open source software package that is well supported by the community and by the Ettus Research Group and the N200 SDR.  GNURadio also comes with what is known as GNURadio Companion or GRC.  This program provides us with a GUI interface and allows for the drag and drop of blocks that represent certain functions that can be used with the SDR.  GNURadio and GRC use Python as it's main scripting language and GNURadio uses C++ code for directly accessing the hardware.  The hardware interface for the N200 is provided by Ettus as well.  They provide drivers that allow GNURadio to talk to their hardware.  Like GNURadio, these drivers are also available to all platforms.  In addition, Ettus has released these drivers to the open source community.

GNURadio has been ported to several platforms including Windows, Mac OS X and Linux.  Linux is by far the most popular platform to work on and most of this thesis research was done within the Linux environment.  Testing was also done though with a MacBook Pro running OS X 10.9.  The OS X implementation is fairly well supported and is installed through MacPorts.  

Through GNURadio, we can now write code that will take the data given to us from the SDR and manipulate the signal as we need to mimic a radiometer.  The power detection, filtering and recording of this data is all done through GNURadio.  This also means that we are shifting more of the computational power done on the signal from the FPGA to a computer running GNURadio.  This is different from the current ISU radiometer where the FPGA did almost all of the work and then simply sent the data to a computer to be stored.  The benefit of this however is that components of the radiometer can be updated by simply changing things in GNURadio and does not require reprogramming the FPGA.  It does however mean that the host computer must be powerful enough to handle the signal and specifically the large bandwidth that we wish to send to it.  

\subsection{RF Front End}
The RF Front End used for the work in this thesis uses components that were originally used in the ISU Radiometer.  This RF front was designed for use in the 1.4 GHz protected frequencies for remote sensing.  

{\begin{figure}[h!tb] 
\centering
\includegraphics{Images/IMG_0027}
\isucaption{Inside the ISU Radiometer showing the FPGA, micro-controller and one of the RF front ends}
\label{ISU_Rad}
\end{figure}
}
The RF front end uses a 3 stage Low Noise Amplifier (LNA) to amplifier the noise while keeping the noise contributed to the system as low as possible.  As with any radiometer, the first LNA is the most critical as it contributes the most to the overall system noise temperature.  For this reason the University of Michigan used a LNA that did not have a large gain but had a very low noise figure. The second and third LNA has higher gain values at the cost of a higher noise figure.  However, since they are further down the chain, they do not contribute as much to the total system noise.  The reason for this is further explained in chapter 3.  

%A driving force behind this thesis research was due to the fact that the digital side of the ISU radiometer was not functioning.  Problems were developed with the FPGA board which resulted in the board being sent back to the University of Michigan several times.  In addition, the micro-controller that stored the data from the FPGA and also controlled various functions within the radiometer also started to malfunction.  In the end, the radiometer was deemed unusable and it was determined that this would be a good time to re-evaluate the current radiometer and see what improvements could be done.  

%Since the current radiometer was very much like a SDR, it was determined to look into the feasibility of using a commercial off the shelf SDR and using it in a radiometer application.  To our knowledge this has not been attempted before and it was determined this would serve as a good thesis platform.

It was assumed in the beginning that the RF front end of the existing radiometer was still functioning properly.  Additional tests on the radiometer did show some intermittent problems which did result in some complications in testing.  These will be explained in more detail later in this thesis.  However, the use of the SDR did help in seeing some of these problems which added an additional benefit to using a SDR to the more traditional radiometers.  

\subsection{Related Works}
As mentioned before, software defined radios have been used in a number of applications.  While as far as the author has determined, this is the first application of using an off the shelf software defined radio for a radiometer in remote sensing for soil moisture measurements, there has been similar applications done.  The closest application that has been found is with radio astronomy.  Radiometers used in radio astronomy is nothing new and has been used for quite some time[\cite{Ohm}]. by There are many similarities between radio astronomy and remote sensing of the ground.  Both are using a radiometer to listen to a source of interest.  In radio astronomy the basic principal is that a "hot" source such as a star will produce more noise than the cooler background of space.  In remote sensing we are looking at the overall change of the source to determine it's characteristics.  In both cases we are measuring the total power of the noise and based on that information we can determine some properties of the source we are looking at.

The Shirleys Bay Radio Astronomy Consortium (SBRAC) located in Smiths Falls, Ontario is currently using a USRP software defined radio in conjunction with GNURadio.  SBRAC has successfully used this configuration to obtain radio astronomy data by looking at the hydrogen line at 1420.4058 MHz [\cite{Leech2007}].  The person in charge of this facility, Marcus Leech, contributed software to the GNURadio specifically for radio astronomy applications.  It was this software branch that was used as the base for the GNURadio program that was used in this thesis.  Marcus Leech continue to contribute to the GNURadio community and continues to provide support for these functions as well[\cite{Leech}].

Another example of a software defined radio used as a radiometer is from students at the University of Illinois and Grand Valley State University that built a software defined radio to listen to emissions from Jupiter[\cite{Behnke}].  This software defined radio was built using an Analog Devices AD9460 and a Xilinx Spartan-3E-500 FPGA to build the SDR itself.  A RF front end was also built to filter and amplify the signal coming into the SDR.  Finally, this group also used GNURadio to interface to and talk to the SDR and used both Python and wxGUI to build a working interface.  The students reported that the SDR radiometer worked very well and was able to do so at a fairly low price point.

It should be noted that much of the related works found worked with using a radiometer for astronomy or for looking at the sky.  For the a soil moisture radiometer however, we are looking at the ground and we are using the radiometer for soil moisture instead of measuring stars and other points of interest in the sky.  While the fundamentals is the same for either radiometer, some adjustments need to be made due to the fact that a radiometer looking at the sky often sees a "cool" brightness temperature whereas a radiometer looking at the ground sees a much "warmer" brightness temperature.  This is due to the albedo of the Earth and the fact that it has a much warmer noise temperature then what you find with radio astronomy[\cite{Tiuri}].

