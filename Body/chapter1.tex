% Chapter 1 of the Thesis Template File
\chapter{THESIS INTRODUCTION}

%This is the opening paragraph to my thesis which explains in general terms the concepts and hypothesis which will be used in my thesis.

%This section will cover the basic and general concepts of my thesis.  This will be a high level approach to the problem and to the proposed solution.  In addition, this section gives a general background to the project

%Chapter 1:  Introduction:
%1) Motivation
%	a) 2-3 sentences on why the felid of Remote sensing important: examples from past and present. [refs]
%	b) 2-3 sentences on what radiometers are (with respect to remote sensing) and examples of how they have been used for remote sensing from the past and more important presently [refs]
%2) Problem
%	a) What aspects of traditional radiometers are limiting their ability to impact society? [refs]
%	b) If these limitations were removed/mitigated, then what additional things could be done with radiometers? [refs or places were radiometers could be useful]
%	c) What high-level approach are you presenting to help address these limitations?  [refs]
%3) Contributions:  2-3 sentences summarizing your contributions (for your thesis you should be able to identify 2 - 3 contributions)
%4) Summary of the organization for the remainder of your thesis

\section{Introduction}
The goal of this thesis is to explore the use of a software defined radio that can function as well as or better than current radiometers.  In addition, we aimed to develop a radiometer that is more flexible than most radiometers and still maintain the accuracy and stability of a traditional radiometers if not exceed these specifications.  A secondary goal was to use off the shelf components that are generally more accessible and often less expensive.  This allows radiometers to be more accessible to a wider scope of researchers in this field.  And finally a tertiary goal was to ensure that the system as a whole is easy to use.  This ties to our secondary goal of making radiometers more accessible to a wider range of researchers and research topics.

This thesis looks to explore the following questions: (1) Can we use an off the shelf SDR along with GNURadio to recreate a radiometer in software that is easy to use and cost effective?  (2) If so, what performance can we get from the system?  (3) What benefits do we gain (if any) from using a SDR from a more traditional radiometer? The results of this research and experimentation are the subject of this thesis.

\section{Motivation}
Remote sensing allows us to collect information about the world that is around us.  This information can give us valuable information such as vegetation health, soil moisture, ocean salinity, and astronomy.  

Radiometers are radio receivers that simply listen to and record the amount of power received.  However, the power received is not a coherent signal, instead it is the amount of noise the radiometer sees.  Radiometers, at the basic level, listen to noise that is generated naturally from a source.  These sources can vary and the applications vary as well.  Some examples of radiometer applications have been in evaluating soil moisture content, ocean salinity levels, and celestial objects[\cite{ulaby2014}].

Using radiometers for remote sensing has several advantages as we are able to look beyond certain obstacles that would otherwise be blocked in other remote sensing methods such as photography.  Radiometers have been used in the past for primarily astronomy in scanning for celestial objects that may be hidden from visual telescopes.  More recently radiometers have now been focusing on Earth and helping scientists to better understand the water cycle on Earth by monitoring our ocean salinity and soil moisture.  

\section{Problem Statement}
Radiometers have proven to be an excellent tool in remote sensing and have been used in a variety of research applications.  While radiometers have been around for quite some time, the equipment and method of implementing a radiometer has not changed much over the last fifty years.  However, radiometers have not seen much widespread use.  This largely due to the fact that there are very few commercially built radiometers, most radiometers are built to fit a specific purpose or research area.  This is in part due to that different tasks may require changes to the radiometer hardware.  This results in radiometers that are often expensive and require a certain amount of upkeep.  

If though, a radiometer can be built with commercial off the shelf components and also be flexible to adapt to different research or experimental needs, then that type of radiometer will be more desirable and will reduce costs by using available COTS solutions.  

The solution to this problem, as outlined in this thesis, is to create a software defined radiometer that digitizes the incoming radio frequency information as soon as possible and then process this information in software instead of using hardware.  In addition, we will use existing hardware and tools that are commercial available and will be lower in cost or free to the user.  

\section{Contributions}
Contributions from the work performed for this thesis include a development of a software defined radiometer that is capable of performing as a radiometer and meets or exceeds the performance of a traditional radiometer.  This includes source code written in the open source software defined radio framework known as GNURadio.  Additional contributions include analysis tools used to analyze data obtained from the software defined radiometer.  Finally contributions were made in using a different technique to mitigate interference that impacts the radiometer readings.  This is accomplished differently from other methods because unlike other radiometers we have both amplitude and frequency information to identify and mitigate this interference.

\section{Thesis Organization}
This thesis is organized to first introduce the user to the background and short introduction to both radiometers and to software defined radios.  Next, a more detailed explanation of the theory that allows for a software defined radio to emulate a radiometer in software followed by details on how key components of a radiometer are implemented in software.  We then examine experimental data that was collected to confirm our hypothsis of software defined radiometer and explore advantages a software defined radiometer has over a more traditional radiometer.  Finally we will analyze our results obtained from our experimental data and ensure it meets our expectations for a software defined radiometer.

%----------------------------------------------------------
% End of Chapter 1.  Anything below this is extra information
