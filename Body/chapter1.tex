% Chapter 1 of the Thesis Template File
\chapter{INTRODUCTION}\label{ch:intro}

%This is the opening paragraph to my thesis which explains in general terms the concepts and hypothesis which will be used in my thesis.

%This section will cover the basic and general concepts of my thesis.  This will be a high level approach to the problem and to the proposed solution.  In addition, this section gives a general background to the project

%Chapter 1:  Introduction:
%1) Motivation
%	a) 2-3 sentences on why the felid of Remote sensing important: examples from past and present. [refs]
%	b) 2-3 sentences on what radiometers are (with respect to remote sensing) and examples of how they have been used for remote sensing from the past and more important presently [refs]
%2) Problem
%	a) What aspects of traditional radiometers are limiting their ability to impact society? [refs]
%	b) If these limitations were removed/mitigated, then what additional things could be done with radiometers? [refs or places were radiometers could be useful]
%	c) What high-level approach are you presenting to help address these limitations?  [refs]
%3) Contributions:  2-3 sentences summarizing your contributions (for your thesis you should be able to identify 2 - 3 contributions)
%4) Summary of the organization for the remainder of your thesis

This thesis explores using software defined radio (SDR) technology to develop a radiometer that has performance on par with that of traditional radiometer.  In addition to demonstrating a SDR-based radiometer can achieve similar performance, in terms of sensitivity and stability, to a traditional radiometer, it is shown that the inherent flexibility of SDR technology allows implementing functionality beyond what a traditional radiometer typically provides.  Furthermore, by reducing cost and increasing flexibility, SDR technology may become an attractive path for broadening the accessibility of radiometry to the general research community.

\emph{Motivation.}  Remote sensing refers to recording, observing and perceiving objects or events that are far away (i.e. remote)[\cite{weng2012}].  Since the object of interest is remote, we cannot physically interact with it using local measurement methods such as placing sensors or probes on the object.  Remote sensing can be accomplished in a variety of ways and the following lists the basic approaches:

\begin{enumerate}
\item Visible light: Photography and Photogrammetry,
\item Thermal, Far infrared: Thermal Infrared Radiometry
\item Laser distance measurement: Lidar,
\item Radio Frequency (RF): Radiometry.
\end{enumerate}

Each of these methods has an associated set of pros and cons, thus remote sensing often uses a combination of methods to paint a complete picture of an object.  This thesis focuses on the apparatus used in radiometry to capture RF signals, called a radiometer.

Radiometry can be broken in to two methods; active or passive.  An active system is one in which a radio frequency signal or pulse is generated and transmitted to the object of interest.  The reflection of this RF signal, or in some cases lack of reflection, gives us information about the object.  A passive system is one in which no RF signal or pulse is generated.  Instead, this type of radiometer simply listens to the RF energy that is naturally generated by the object or that may be reflected from another source, such as the Sun.  Radiometers have been focused on Earth for such purposes as helping scientists better understand its water cycle by monitoring ocean salinity [\cite{Hardy}] and soil moisture [\cite{Liu}].  Radiometers such as these are already in service on satellites such as the Soil Moisture and Ocean Salinity (SMOS) satellite [\cite{McMullan}] launched by the European Space Agency (ESA).  Additional applications of radiometry include: assessing vegetation health and observing celestial objects[\cite{ulaby2014}].

\emph{Problem Statement.}  While radiometers have proved to be an excellent tool for remote sensing and have been used in research applications for over fifty years, they have not made it into wide spread use.  This is due to the fact that many traditional radiometers have the following hurdles:

\begin{enumerate}
\item They are expensive,
\item They require advanced knowledge to implement and use,
\item They are typically built and designed for a custom application.
\end{enumerate}

This thesis aims to help address each of these hurdles as follows.  The use of commercial of the shelf (COTS) parts and solutions will help reduce cost and the need for a custom design.  Software will be used to define key components of the radiometer that have traditionally been implemented in hardware and to ease adaptation to multiple applications.  A user friendly graphical user interface (GUI) will help reduce the advanced knowledge required to implement and use the radiometer. 

\emph{Contributions.}  The primary contributions of this thesis are:

\begin{enumerate}
\item Development of a software defined radio-based radiometer
\item Python based scripts for analyzing data generated by the radiometer
\item A Radio Frequency Interference (RFI) mitigation technique implemented using software defined radio technology
\end{enumerate}

\emph{Organization.}  The remainder of this thesis is organized as follows.  Chapter \ref{ch:relatedworks} gives a discussion of related works.  Chapter \ref{ch:background} gives background on traditional radiometers and on software defined radios.  Chapter \ref{ch:implementation} presents details on how software defined radio technology was used to implement a radiometer.  Chapter \ref{ch:exp_design} describes the experimentation setup used to verify and evaluate the operation of the implemented radiometer.  Chapter \ref{ch:results} examines the results obtained from our performance evaluation experiments.  Chapter \ref{ch:conclusion} concludes this thesis and outlines avenues of future work.

%----------------------------------------------------------
% End of Chapter 1.  Anything below this is extra information

%This thesis looks to explore the following questions: (1) Can we use an off the shelf SDR along with GNURadio to recreate a radiometer in software that is easy to use and cost effective?  (2) If so, what performance can we get from the system?  (3) What benefits do we gain (if any) from using a SDR from a more traditional radiometer? The results of this research and experimentation are the subject of this thesis.