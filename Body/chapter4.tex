% Chapter 4 from the standard thesis template
%   that contains an adv. example table and figure.

%Chapter 4 discusses the Experimental setup and experiments used and how I evaluated the performance of the radiometer.

\chapter{EVALUATION AND EXPERIMENTAL SETUP}
Testing and verification ensures that the new components that we have added to the system are working as intended and has not caused a negative impact on the overall system performance.  To do this, several tests were conducted and verification was obtained by using a well known method of detecting power in a radiometer, a square-law detector.  

Testing began with the square-law detector as this was one of the first components that was obtained.  Next, testing was done with the the software defined radio as a whole.  Finally a cold bath test was done with both the software defined radio and with the square law detector as a system to verify both functionality and to compare the results from both devices.  

Additional testing was also performed to test additional functions that are not found on a typical radiometer.  This test involved injecting a signal and having the software defined radio remove the offending signal.  A comparison was then performed to show the difference between the SDR radiometer which is able to remove the offending signal and the square-law detector which is not able to remove it unless additional hardware filters are placed in front of it.

In addition to these tests, a few real world tests were also done in the Spring of 2012 and the Spring of 2014.  These tests were conducted by Dr. Brian Hornbuckle's E E 518 class was conducted as part of their lab requirement for the test.  These test results can be found in Appendix 2.

\section{Square-law Detector}
To verify the results of the information that the software defined radio is obtaining, a square-law detector is used to measure the power of the incoming signal in parallel to the software defined radio.  The incoming signal is split using a power divider so that the signal will be the same to both devices with the exception of the 3 dB plus insertion loss the power divider adds.  This allows us to verify the software defined radio with a proven system.  Square-law detectors have traditionally been used in radiometers and have been proven to work in radiometer applications.  They are also a very simple device which means there is little that can go wrong with using them.

\subsection{Analog Devices ADL5902}

The square-law detector that we obtained is the Analog Devices ADL5902.  The ADL5902 is a true rms responding power detector that has a square law detector, a variable gain amplifier and an output driver. It also has a temperature sensor and will compensate for temperature variations.  The output driver allows for the small signal from the square law detector to be amplified to a level that most analog to digital devices can detect.  It should be noted that this is not an amplifier for the incoming RF signal, just to amplify the small signal from the diode.  This driver however does have low noise and has a noise output of approximately $25nV/ \sqrt{Hz}$ at 100 kHz.  The ADL5902 operates from 50 MHz to 9 GHz and in most cases can detect down to $-60$ dBm.  This works well in our application since the radiometer operates at 1.4 GHz and after the amplification stage we usually see between $-40$ to $-30$ dBm of power.  

{\begin{figure}[h!tb] \centering
\includegraphics[width=\textwidth]{Images/adl5902.jpg}
\isucaption{An image of the ADL5902 square-law detector used in these experiments}
\label{ADL5902}
\end{figure}
}

The specifications of the ADL5902 can be seen in Table~\ref{ADL5902_data} shown below.

\begin{table}[h!tb] \centering
\isucaption{ADL5902 Specifications}
\label{ADL5902_data}
% Use: \begin{tabular{|lcc|} to put table in a box
\begin{tabular}{lcc} \hline
\textbf{Parameter @ 900 MHz} & \textbf{Value} & \textbf{Units} \\ \hline
Frequency Range & 50 to 9000 & MHz \\
Dynamic Range & 61 & dB \\
Minimum Input Level, $\pm 1.0$ dB & 60 & dB \\ 
Maximum Input Level, $\pm 1.0$ dB & 1 & db \\
Logarithmic Slope & 53.7 & mV/dB \\ 
Output Voltage Range & 0.03 to 4.8 & V \\ \hline
\end{tabular}
\end{table}

The ADL5902 outputs an analog signal that falls between 0.03 volts and 4.8 volts.  It outputs a change of 53.7 millivolt per dB detected by the ADL5902.  

\subsection{Data acquisition and storage}

In order to analyze the data, a method was developed to acquire the data and store it for later use.  For the N200 software defined radio, this is done automatically by the GNURadio program.  Both the complete signal and the power information is stored to a file for later analysis.  The square-law detector however outputs information as an analog voltage that is linearly proportional to the RF power measured.  This required a system that can capture the analog signal from the ADL5902 and then send the data to be stored.  This was accomplished by using a National Instruments USB-6009 data acquisition unit.  This unit has 8 analog inputs that can sample up to 48 ksps with a resolution of 14-bits.  This was more than adequate for our needs.  To use the USB-6009 a fairly simple Lab View program was created.  This program retrieved the information from the USB-6009 and stored the data in both Labview's binary format and in a more human friendly ASCII format.  The USB-6009 then connects to a host computer through the USB interface.  This made obtaining the data and using device fairly straightforward to use.

\subsubsection{Labview Acquisition program}

The Labview program was created to talk to the USB-6009 and then store the data.  In addition, it is able to display in real time the current readings coming from the USB-6009.  This was beneficial during testing of the program but is also good information to have while running an experiment.  

{\begin{figure}[h!tb] \centering
\includegraphics[width=\textwidth]{Images/labviewGUI.png}
\isucaption{A screenshot of the Labview GUI interface}
\label{labviewgui}
\end{figure}
}

The Labview program uses National Instruments DAQ assistant which allows for quick configuration and setup for the computer to talk to a number of NI devices.  Labview also includes blocks that allows us to easily record the data to a file.  These blocks made up most of the program and resulted in a program that was quickly made.  

While the USB-6009 allows up to 48,000 samples per second, we don't need such a high rate.  A rate of 1,000 samples per second was determined to provide more than enough samples of data.  With such a high rate of samples however, the resulting output will be noisy due to the natural noise found in the RF signal.  Like the software defined radio, we want to filter this noise as well to produce a smoother output.  This was accomplished using a filter block in Labview to help reduce the noise.

{\begin{figure}[h!tb] \centering
\includegraphics[width=\textwidth]{Images/labview-diagram.png}
\isucaption{A screenshot of the Labview block diagram}
\label{labviewblock}
\end{figure}
}

This program allowed for the data from the square law detector to be easily recorded.  Using Labview also allows us to customize the interface more to our liking and other features such as adding calibration information can also be added as well.  

\subsection{Tests on the ADL5902}
To test the ADL5902 a signal generator was used that had a controlled output.  The specific signal generator that was used was an older model that allowed us to change the output in 10 dBm increments.  The signal generator was also configured for 1.4 GHz as that is the frequency the ISU RF front end is configured to listen at.  Ideally a noise generator would have been desirable, however a noise generator with adjustable power output could not be located on campus.  

The ADL5902 is available from Analog Devices in an evaluation board.  This board pairs the ADL5902 with a AD7466 12-bit analog to an analog to digital converter.  This board can then be mated with Analog Devices BlackFin processor which acts as a USB gateway for the AD7466 data.  A test program written in LabView is also provided as well.

The test program provided by Analog Devices allows us to query the ADL5902 and record the raw ADC value.  The test program also allows us to enter in the frequency used during testing and the temperature during the test.  The test program also allows us to calibrate the system as well.  All of the data is then stored into an Excel spreadsheet which can be accessed later.

For this test we used the signal generator set to 1.4 GHz and started at $-60$ dBm for the output signal.  This was selected as this is the lowest the square law detector can detect.  The output power was then incremented on the signal generator in 10 dBm steps.  There was no change to any other parameter.  This was done up to 0 dBm.  Any higher and there was risk of damage to the ADL5902.  This test was then repeated several times and was done with the signal generator stepping up from $-60$ dBm to 0 dBm and from 0 dBm to $-60$ dBm.  

The data collected was then graphed using Excel.  The graph shown in~\ref{adl5902_linear}
Shows that the ADL5902 has a linear output and matches the expected value based on the input.  

\begin{figure}[h!tb] \centering

\includegraphics[width=\textwidth]{Images/Linearsquarelaw}

\isucaption{Graph showing the linearity of the ADL5902}
\label{adl5902_linear}
\end{figure}

Once it was confirmed the ADL5902 does in fact have a linear response, we then looked at ways to work with the ADL5902.  The graph above was generated using Analog Devices program that talked to the Blackfin processor on the daughter-board.  However, the information for talking to the Blackfin is proprietary.  Therefore the daughter-board was removed for future tests and instead we used the USB-6009 data acquisition board to read the raw analog voltages from the ADL5902.  This information was then stored later for additional analysis.  

\section{Software Defined Radio Experiments}
Once it was confirmed that the square-law detector was working within the specifications that were given, testing then moved to the software defined radio.  Once the Python program was established for replicating a total power radiometer in software, this was then loaded into GNURadio and used to control the N200 SDR.  

Before testing the program with the N200, testing was done with the built in noise generator in GNURadio.  This was used to test it's ability to measure small changes in noise.  This simulated noise verified that the program written was able to detect changes in noise power using a simulated Gaussian source.  It was also desired to also use a hardware based noise generator, however a suitable noise generator could not be located on campus.

{\begin{figure}[h!tb] \centering
\includegraphics[width=\textwidth]{Images/noisesrc_radiometer.png}
\isucaption{A screenshot of the GNURadio program using a noise generator block.}
\label{noise_test}
\end{figure}
}

After this was done additional tweaks were made to both the GUI interface and to the program that performs the calculations for the total power radiometer.  The essential components were moved to a custom block which helped to clean up the block diagram.  Details on how this works was discussed in chapter three.

Three experiments were conducted to verify the operation of the software defined radio.  One experiment that tested stability will be discussed in chapter 5.  The other two experiments will be discussed in detail here.  These two experiments verified that the software defined radio behaved in a similar fashion to the square-law detector and also demonstrated an additional function that the square-law detector was not able to do.

\section{Experiment 1 - Calibration and Normal Radiometer Operation}

To fully test both the total power radiometer in the software defined radio and the square-law detector, a cold water bath experiment was established to verify that both the square-law detector and SDR was able to measure real world changes in the noise.  In addition, this would also give us a calibration line to to calibrate the radiometer.

\subsection{Laboratory Setup}
For this experiment the, ISU radiometer was setup in the Make to Innovate lab located in Howe Hall at Iowa State University.  This lab was picked as it has a an electronics area and has test equipment needed for certain experiments as well as the needed power supplies to power the equipment.  To measure the change in noise, a 50 ohm matched load was attached to the input of the LNAs and this output was then feed to a power divider which divides the signal between the software defined radio and the square-law detector.  This does add a 3.1 dB loss to both devices, however this was acceptable for these experiments.

\begin{figure}[h!tb] \centering

\includegraphics[width=\textwidth]{Images/exp1_setup.jpg}

\isucaption{An image of the typical lab setup used to test the software defined radiometer}
\label{LabSetup}
\end{figure}

The 50 ohm load was then submersed into a hot or cold bath.  These baths were temperature controlled first by using Liquid Nitrogen (LN2) which is known to boil at 77 K and an ice water bath which is known to be at 273.15 K.  The temperature of these could easily be monitored and maintained.  The load was submersed in each bath for a minimum of 2 minutes to allow for it to reach the same temperature as the bath.  The noise measured was then recorded using GNURadio with the N200 SDR and with the USB-6009 to record the square-law data.  In addition to the total power measurements, the raw I-Q data was also recorded.  This allowed us to replay the experiment through GNURadio for further study.
\subsection{Test Results}

The first data we will look at with be with the software defined radio.  The SDR records the total power measurements to a binary file that either Matlab or Python can read.  In our results, we used iPython Notebook to read and generate the graphs used in this thesis.  We will begin by looking at the raw total power readings which we will call rQ values.  These are the uncalibrated total power readings from the software defined radio.  

\begin{figure}[h!tb] \centering

\includegraphics[width=\textwidth]{Experiments/Exp1/rqvstime_annotate.pdf}

\isucaption{Graph of the un-calibrated rQ values of Experiment 1}
\label{SDR_rQ}
\end{figure}

Figure \ref{SDR_rQ} shows the total power reading versus time and is also marked when the matched load was submerged in Ice Water, LN2, or a hot water bath.  Since we know what the temperatures are for the ice water and LN2, we can now calibrate these readings to a noise temperature reading.  This is done by reading in a calibration file we have stored in csv format and performing linear algbra to solve the slope of the line.  This was done in our iPython Notebook using the following code.
\lstset{language=Python}
\begin{lstlisting}[frame=single,keywordstyle=\color{blue}]
a = numpy.array([[rQ_values[0],1.0],[rQ_values[1],1.0]],numpy.float32)
b = numpy.array([temp_values[0],temp_values[1]])
z = numpy.linalg.solve(a,b)
\end{lstlisting}

Now that we have our calibration points, we can now re-graph this data but now have it reference the noise temperature. 

\begin{figure}[h!tb] \centering

\includegraphics[width=\textwidth]{Experiments/Exp1/sdr_calibrated_color.pdf}

\isucaption{Graph of the calibrated noise temperature of Experiment 1}
\label{SDR_Calibrated}
\end{figure}

Now that we have looked at the software defined radio data, we want to look at the square-law detector with the end result of comparing the two.  The square-law detector gives us power information as a voltage, so once again we will need to calibrate this to the known temperature references.  However, before that we need to do one other step with the data.  Unlike the software defined radio, there is no filter or integrator to help smooth out the data, so the square-law data is very noisy.  Our first step will be to filter the data.  Figure \ref{X2_Raw} shows our raw data that we get from the square-law detector.

\begin{figure}[h!tb] \centering

\includegraphics[width=\textwidth]{Experiments/Exp1/noisy_voltage.pdf}

\isucaption{Raw and noisy graph from the square-law detector used in Experiment 1}
\label{X2_Raw}
\end{figure}

Once again we can use Python to process this information and specifically we can use SciPy which includes several useful signal processing modules.  For our use, we will use a low pass filter to clean up the signal.  The following code allows us to do just that.

\begin{lstlisting}[frame=single,keywordstyle=\color{blue}]
from scipy import signal
N=100
Fc=2000
Fs=1600
h=scipy.signal.firwin(numtaps=N, cutoff=40, nyq=Fs/2)
x2_filt=scipy.signal.lfilter(h,1.0,x2_voltage)
\end{lstlisting}

Now we can re-graph this with a better with the filter in place.

\begin{figure}[h!tb] \centering

\includegraphics[width=\textwidth]{Experiments/Exp1/x2_filter.pdf}

\isucaption{Filtered data from the square-law detector used in Experiment 1}
\label{X2_filter}
\end{figure}

Using the same technique as earlier, we can now calibrate the raw voltages from the square-law detector to the noise temperature.

\begin{figure}[h!tb] \centering

\includegraphics[width=\textwidth]{Experiments/Exp1/x2_calibrated.pdf}

\isucaption{Calibrated data from the square-law detector used in Experiment 1}
\label{X2_Calibrated}
\end{figure}

We now want to compare both the Software Defined Radio and the square-law to make sure that they match up.  Since both the SDR and the square-law are now calibrated to a noise temperature, we can easily graph both of the data and compare to see how well they match up.

\begin{figure}[h!tb] \centering

\includegraphics[width=\textwidth]{Experiments/Exp1/x2_SDR_Calibrated.pdf}

\isucaption{Figure showing both the SDR and square-law noise temperature data in Experiment 1}
\label{X2_SDR_Both}
\end{figure}

We can see in Figure \ref{X2_SDR_Both} that both the software defined radio and the square-law detector match up very nicely.  This shows that both the square-law detector and the software defined radio agree when properly calibrated.  This verifies that the software defined radio can indeed operate as a total power radiometer and the data we obtain from this setup agrees with an analog and more traditional radiometer.
\subsection{Application with Soil Moisture Readings}

\section{Experiment 2 - Interfering Signal Mitigation}
Experiment 2 is used to show additional benefits of using a software defined radio compared to a more traditional radiometer.  One benefit of using a SDR is that key components such as filters are created in software instead of using hardware.  This means that filters can be created on the fly and can be adjusted simply by updating the software.  For this experiment we will use both the Software Defined Radio running GNURadio with the radiometer firmware and a square-law detector that is connected to a data acquisition unit.  The experiment uses a 50 ohm matched load that is attached to the radiometer.  This matched load is then submerged into two different baths, a Liquid Nitrogen at ~ 77 K and an Ice bath at ~ 273.15 K.  Finally the HackRF One is used to generate the offending signal.  This signal is generated within the 1400 to 1425 MHz that the radiometer is configured for.   for an extended period of time.  This was to determine the stability of the radiometer as it is expected that as long as the LN2 is covering the matched load, it should maintain the temperature at 77 K.  

\subsection{Testing Apparatus}
To run this test, we need to have some equipment for this.  First and foremost, we need a radiometer.  For this test we used the ISU Radiometer front end, with the LNAs and bandpass filters in place.  The noise diode was turned off for these experiments.  A fifty ohm matched load was then attached to low loss coax and represented our source to the radiometer.  Finally, the output of the radiometer was run to the Ettus N200 software defined radio instead of running the on-board Analog to Digital converter and FPGA.  The data from the N200 was then sent to a computer running GNURadio and the custom software that I had written.  This data is sent to the computer through a 1 gigabit Ethernet connection due to the large bandwidth coming from the N200.

\subsection{Calibration and verification experiment}
The first test run was conducted to help verify that the radiometer was operating correctly and to start to build up data for calibration.  This test was conducted by placing the matched load into the liquid nitrogen bath, which has a boiling temperature of 77 K and then putting the matched load into a ice water bath which has a temperature of 273.15 K.  This would give us two points to figure out a calibration point.

During this experiment both the square law data and data from the software defined radio was recorded.  This information was then feed into a Matlab script that is able to read the data files, perform the needed calculations and then plot the data.  

\begin{figure}[h!tb] \centering

\includegraphics[width=\textwidth]{Experiments/Exp1/x2_SDR_Calibrated.pdf}

\isucaption{Graph showing both the square-law detector and the software defined radio total power readings.}
\label{lab1_x2_n200}
\end{figure}

Data from the N200 was stored to file using the GNURadio file sink which stores the data in a binary format.  The square-law detector was feed into a LabView DAQ which then stored the data in both an ASCII and LabView's TDMS file format.  In addition the LabView program displayed the square-law data in real time

\begin{figure}[h!tb] \centering

\includegraphics[width=\textwidth]{Images/labviewx2_lab0.png}

\isucaption{Screen shot of the Labview program capturing total power information from the radiometer}
\label{labview_tpr}
\end{figure}

This experiment showed that both the square-law detector and the software defined radio was able to detect changes in the noise temperature.  It was also able to show that once calibrated, both showed identical results.

\section{Testing with an injected noise source}
The addition of an unwanted signal can be a determinant to the radiometer and has an adverse affect on how the radiometer operates.  In today's world though, it is getting more difficult to control intentional radiators as the RF spectrum becomes crowded with more devices.  This problem becomes even a greater problem with radiometers used in orbiting spacecrafts as they are able to "see" large areas[\cite{DeRooRFI}].  Even though the band we are working in of 1.4 GHz is an internationally protected frequency, there can still be both intentional and unintentional radiators that cause interference.  This has been seen in some of the SMOS satellite data.

\begin{figure}[h!tb] \centering

\includegraphics[width=\textwidth]{Images/interfering_signal_edit.png}

\isucaption{Image of the interfering signal appearing on the spectrum display of the SDR Radiometer.}
\label{inter_signal}
\end{figure}

RFI detection and mitigation is not a new topic in radiometry and there have been other methods in both the detection and mitigation of these signals[\cite{Forte}][\cite{McIntyre_RFI}][\cite{DeRoo}][\cite{Ellingson}].

This test was designed to test a problem that a software defined radio would be to cope with while a square law detector would not be able to cope with it.  This test injected a known signal at 1.406 GHz to interfere with the normal operation of the radiometer.  In this test, the square law detector, since it is a wide band device, would not be able to accurately measure the change in the total power of the signal.  However, the SDR is able to create a digital filter to filter out the offending signal.  Since this is done in software, the SDR is able to adapt to changes much faster than an analog radiometer.

For this test however, we kept the experiment simple and fixed the interfering signal to a known frequency and amplitude.  The filter was then designed with these parameters in mind.  This type of radiometer operating in a real world environment would need some additional programming to identify the offending signal and then mitigate.  However this test shows that the SDR is capable of responding to an offending signal.

\subsection{Test setup}
To test this theory, a similar setup was used as with previous tests.  The ISU RF front end was once again used to amplify the signal.  One difference however is that a power combiner, which is the same as a power divider, was used to hook up to a signal generator.  The signal generator then provided the offending signal.  By using the signal generator we can control how much power and what frequency the offending signal is at.  In this case we are performing this in a controlled setup and would know ahead of time where the offending signal is.  In the future the SDR defined could identify the offending signal and then filter it out.

Like the other tests, a power divider is used to split the signal after the ISU RF front end which was then feed to a square-law detector and then the N200 SDR.  The square-law detector was then hooked up to a National Instruments USB-6009 DAQ.  The DAQ was then feed into a Labview program to be processed and recorded.  The N200 continued to use the GNURadio program that was created to record the total power coming out.  However, it has been modified with a band reject filter to filter the offending signal.

\subsection{Test results}
Results of this test showed two key things with the SDR.  The first being that the SDR is capable of filtering out an offending signal.  This can be seen in figure \ref{filter_on}.

\begin{figure}[h!tb] \centering

\includegraphics[width=\textwidth]{Experiments/Exp4/calib_filtered_both.pdf}

\isucaption{Image of the offending signal being filtered out by the SDR.  It can be seen that the signal is no longer visible.}
\label{filter_on}
\end{figure}

The second result is with the SDR still being able to take total power measurements. 

\begin{figure}[h!tb] \centering

\includegraphics[width=\textwidth]{Images/moneyshot.png}

\isucaption{Image showing both the SDR and square-law detector recording TPR, but with the SDR using a filter to remove the offending signal.}
\label{sdr_x2_filter}
\end{figure} 

\section{Further testing}
The software defined radio radiometer can be used in other configurations as well.  Further testing is needed to compare some of these other modes to a traditional radiometer to verify their operation, but in theory these additional configurations should operate the same in a software defined radio as they do with a traditional radiometer.