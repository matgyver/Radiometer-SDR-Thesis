%Chapter 6 will look at the results from the experiments and how the performance of the new software defined radiometer measured up.

%Chapter 6: Results and Analysis

%Intro paragraph.
%Experiment 1
%	Data collected.
%	Analysis of data: Interpret the data, what was learned from the data
%Experiment 2
%	Data collected.
%	Analysis of data: Interpret the data, what was learned from the data
%Experiment X
%	Data collected.
%	Analysis of data: Interpret the data, what was learned from the data
%Price/Weight/Quality analysis (Traditional Radiometer vs. Software Defined Radiometer)
%Summary paragraph, the summary paragraph the summarized the analysis of the results.

\chapter{RESULTS AND ANALYSIS}\label{ch:results}
This chapter will display the results obtained from the experiments outlined in Chapter \ref{ch:exp_design}.  We will then give an analysis of the data included data interpretation and the information learned from each experiment.  Next we will do an analysis of a traditional radiometer vs our software defined radiometer in terms of price, weight, and quality.  Finally we will summarize our results and analysis from this chapter.

\section{Experiment I - Software Defined Radiometer Verification and Calibration} \label{Exp1_results}
As outlined in Chapter \ref{ch:exp_design}, Section \ref{Exp1}, this experiment is to verify the operation of a software defined radiometer.  This is done by performing experiments that are similar to verification and calibration methods used on a traditional radiometer.  By comparing these results to a square-law detector, receiving the same signal, we can verify the operation of the software defined radiometer.

\subsection{Data Collected}

The first data we will look at with be with the software defined radio.  The SDR records the total power measurements to a binary file that either Matlab or Python can read.  In our results, we used iPython Notebook to read and generate the graphs used in this thesis.  We will begin by looking at the raw total power readings which we will call raw Q values or rQ values.  These are the uncalibrated total power readings from the software defined radio.  

\begin{figure}[h!tb] \centering

\includegraphics[width=\textwidth]{Experiments/Exp1/rqvstime_annotate.pdf}

\isucaption{Graph of the un-calibrated rQ values of Experiment 1}
\label{SDR_rQ}
\end{figure}

Figure \ref{SDR_rQ} shows the total power reading versus time and is also marked when the matched load was submerged in Ice Water, LN2, or a hot water bath.  Since we know what the temperatures are for the ice water and LN2, we can now calibrate these readings to a noise temperature reading.  This is done by reading in a calibration file we have stored in csv format and performing linear algebra to solve the slope of the line.  This was done in our iPython Notebook using the following code.
\lstset{language=Python}
\begin{lstlisting}[frame=single,keywordstyle=\color{blue}]
a = numpy.array([[rQ_values[0],1.0],[rQ_values[1],1.0]],numpy.float32)
b = numpy.array([temp_values[0],temp_values[1]])
z = numpy.linalg.solve(a,b)
\end{lstlisting}

Now that we have our calibration points, we can now re-graph this data but now calibrated as a reference noise temperature. Figure \ref{SDR_Calibrated} shows a calibrated graph of the data in relation to noise temperature.  In addition, we have colorized this graph to show warmer to cooler temps.  This is helpful as we often refer to these as noise temperatures.

\begin{figure}[h!tb] \centering

\includegraphics[width=\textwidth]{Experiments/Exp1/sdr_calibrated_color.pdf}

\isucaption{Graph of the calibrated noise temperature of Experiment 1}
\label{SDR_Calibrated}
\end{figure}

Now that we have looked at the software defined radio data, we want to look at the square-law detector with the end result of comparing the two.  The square-law detector gives us power information as a voltage, so once again we will need to calibrate this to the known temperature references.  However, before that we need to do one other step with the data.  Unlike the software defined radio, there is no filter or integrator to help smooth out the data, so the square-law data is very noisy.  Our first step will be to filter the data.  Figure \ref{X2_Raw} shows our raw data that we get from the square-law detector.

\begin{figure}[h!tb] \centering

\includegraphics[width=\textwidth]{Experiments/Exp1/noisy_voltage.pdf}

\isucaption{Raw and noisy graph from the square-law detector used in Experiment 1}
\label{X2_Raw}
\end{figure}

Once again we can use Python to process this information and specifically we can use SciPy which includes several useful signal processing modules.  For our use, we will use a low pass filter to clean up the signal.  The following code allows us to do just that.

\begin{lstlisting}[frame=single,keywordstyle=\color{blue}]
from scipy import signal
N=100
Fc=2000
Fs=1600
h=scipy.signal.firwin(numtaps=N, cutoff=40, nyq=Fs/2)
x2_filt=scipy.signal.lfilter(h,1.0,x2_voltage)
\end{lstlisting}

Figure \ref{X2_filter} now shows our data that has been filtered by the low pass filter.  This is similar to the low pass filter that is also used by the software defined radio.

\begin{figure}[h!tb] \centering

\includegraphics[width=\textwidth]{Experiments/Exp1/x2_filter.pdf}

\isucaption{Filtered data from the square-law detector used in Experiment 1}
\label{X2_filter}
\end{figure}

Using the same technique as earlier, we can now calibrate the raw voltages from the square-law detector to the noise temperature.  Like the software defined radio data, we can calibrate the voltages from the square-law detector to the physical temperature that our matched load is placed in.  Figure \ref{X2_Calibrated} shows the data from the square-law detector calibrated to our known temperature points.  This now allows us to directly compare the square-law detector to the software defined radio data since we have a common reference point in which to compare the data.

\begin{figure}[h!tb] \centering

\includegraphics[width=\textwidth]{Experiments/Exp1/x2_calibrated.pdf}

\isucaption{Calibrated data from the square-law detector used in Experiment 1}
\label{X2_Calibrated}
\end{figure}

We now want to compare both the Software Defined Radio and the square-law to make sure that they match up.  Since both the SDR and the square-law are now calibrated to a noise temperature, we can easily graph both of the data and compare to see how well they match up.

\begin{figure}[h!tb] \centering

\includegraphics[width=\textwidth]{Experiments/Exp1/x2_SDR_Calibrated.pdf}

\isucaption{Figure showing both the SDR and square-law noise temperature data in Experiment 1}
\label{X2_SDR_Both}
\end{figure}

We can see in Figure \ref{X2_SDR_Both} that both the software defined radio and the square-law detector match up very nicely.  This shows that both the square-law detector and the software defined radio agree when properly calibrated.  This verifies that the software defined radio can indeed operate as a total power radiometer and the data we obtain from this setup agrees with an analog and more traditional radiometer.
\subsection{Application with Soil Moisture Readings}
A common application of radiometers is in the measurement of soil moisture.  All items naturally emit RF energy due to the random excitation by the electrons in the object.  The amount of noise that gets generated varies by temperature, but the amount that reaches the antenna varies by the amount of moisture in the soil.  If we can calibrate the radiometer to two known soil conditions, then we can measure the various levels of soil moisture in the soil.  At this time, we will simply look at the percentage of moisture in the soil, which will vary from zero percent or dry soil to one hundred percent or very wet soil.  The drier the soil, the more thermal noise we receive and the "warmer" the noise temperature.  Wet soil on the other hand attenuates the thermal noise and shows up as a "cooler" noise temperature.  

Using the Software Defined Radio, we can set up two methods to visually look at the noise temperature and thus the soil moisture percentage.  Since we do not have an antenna hooked up, we will simulate this by using two reference temperatures.  In this case the ice water bath and the LN2 that we just used and was shown earlier.  

A unique visual aid we can use with GNURadio is a waterfall display.  This display gives us information that includes frequency, amplitude and time.  It is referred to as a waterfall display due to the fact the display continually moves from top to bottom and looks like a waterfall.  Amplitude information is given by mapping the range of amplitudes to a color bar.  Frequency is given on the x-axis and time is on the y-axis.  Figure \ref{LN2_waterfall} shows a screen shot of the waterfall display in the GUI created in GNURadio Companion for this experiment.  This is the same program we have used but with the waterfall display now added.  The data for the waterfall is pulled directly from our source block or in our case the N200 software defined radio.

\begin{figure}[h!tb] \centering

\includegraphics[width=\textwidth]{Experiments/Exp1/LN2_waterfall.png}

\isucaption{Screenshot of the waterfall display used in Experiment 1}
\label{LN2_waterfall}
\end{figure}

Let's take a look at two screen shots of the waterfall.  We will put them side by side so we can better compare the display when the thermal load is in the ice water bath and when the load is in LN2.

\begin{figure}[h!tb] \centering

\includegraphics[width=\textwidth]{Experiments/Exp1/waterfall_side.pdf}

\isucaption{Side by side comparison of the waterfall display for Experiment 1}
\label{side_waterfall}
\end{figure}

In figure \ref{side_waterfall} we can see that the ice water screen shot appears warmer than the right side screen shot which appears cooler.  There is a limitation in GNURadio and the waterfall display that limits what the range for the power readings are.  The overall power that we see only changes by about 3 dB and our current range in the waterfall display is set to 10 dB.  If we could reduce the range, the color change would be even greater and more pronounced.  However, this does show that a change can be seen visually with color to indicate the noise temperature.  

If we assume that our LN2 is dry soil and that our ice water bath is wet soil, we can now interpolate the data to this scale and show the information we obtained in Experiment one as both a noise temperature and soil moisture.  Figure \ref{SDR_soil} shows the same data we looked at earlier but we have now added a scale to show soil moisture as a percentage.

\begin{figure}[h!tb] \centering

\includegraphics[width=\textwidth]{Experiments/Exp1/sdr_soilmoisture.pdf}
\isucaption{Plot of the noise temperature of Experiment 1 with soil moisture percentage}
\label{SDR_soil}
\end{figure}

While this demonstrates that we can collate our total power readings with a soil moisture percentage, we would use actual field tests to calibrate the radiometer.  In addition, we could also calibrate to soil moisture content instead of a percentage if desired.  Both methods have been done with traditional radiometers[\cite{Jonard}][\cite{Shi}.

\section{Data Analysis}





%Move this section once the stability section is established
To verify stability of the radiometer, we look to see how much change the radiometer records over a relatively long period of time.  To test this a matched load was submerged in a liquid nitrogen bath for an extended period of time, in this case for fifteen minutes.  The readings were then looked at to study the trend of the data.  The data is graphed in figure \ref{Stability}.

\begin{figure}[h!tb] \centering
\includegraphics[width=\textwidth]{Experiments/Exp2/sdr_calibrated_zoom.pdf}
\isucaption{Graph of the calibrated total power over a period of fifteen minutes.}
\label{Stability}
\end{figure}

\begin{figure}[h!tb] \centering
\includegraphics[width=\textwidth]{Experiments/Exp2/calib_vstime_stddev.pdf}
\isucaption{Graph of the calibrated total power with the standard deviation plotted.}
\label{Stability_calib}
\end{figure}

As it can be seen in figure \ref{Stability}, the amount of change over a period of one hour is quite small.  The standard deviation for this sample is 0.09 kelvin.  The $NE\Delta T$ calculated using 10 MHz for the bandwidth, an integration time of 2 seconds and with our sample at 77 Kelvin is calculated to be 0.10 Kelvin with a system temperature of 350 Kelvin.  Therefore, our system is behaving as we expect it to for this stability test.

Figure \ref{Stability_calib} shows a graph of the total readings but with the standard deviation now plotted.  This shows that the system is stable and is operating as expected within our expected $NE\Delta T$.

%---------------------------------------------------------------


Once the experimental data was obtained the next step is to analyze the data and format the data so that it is easy to read and comprehend.  The total power readings and the raw I/Q data generated from GNURadio is stored as a binary file stored in little-endian format.  Total power data is stored as a float values and I/Q data is stored as complex values.  Data from the square-law detector is stored as a comma delimited ASCII file.

Matlab is one tool we can use to process the information that is stored by GNURadio.  Appendix A contains the Matlab source that will read the total power file generated by GNURadio.  It then calculates information such as the NE$\Delta$T and the calibration points based on the user input.  We can also use Matalb to graph this information as well.

While Matlab is one tool, other tools can be used.  Python for example is also capable of reading in these files and when paired with NumPy and SciPy can be used to perform analysis on the data as well [\cite{Uengtrakul}].  In addition, the open source mathematical program Octave should also be able to read and work with these files.  For this thesis both Matlab and Python was used to provide analysis on the data.  

Most of the graphs generated in this thesis was generated using iPython notebooks.  IPython notebooks uses Python but allows it to be executed in a web browser either locally or on a server.  Using iPython notebooks however also allows us to add additional information using Markdown and basic HTML.  This allowed the author to paint a complete picture of the experimental results illustrating pictures of the setup and the code and steps used to analyze the data.  You can find these notebooks on the author's Github site and can use NBViewer to view them.  An example link is \href{http://nbviewer.ipython.org/github/matgyver/Radiometer-SDR-Thesis/blob/master/Experiments/Exp1/radiometer_experiment_1.ipynb}.

{\begin{figure}[h!tb] 
\centering
\includegraphics[width=17cm]{Images/python_gnuradio.png}
\isucaption{A screenshot showing the iPython notebook code and related graphs generated for parsing GNURadio data}
\label{matlab_display}
\end{figure}

\section{Benefits to Software Defined Radio Radiometer}
A study was conducted on what benefits a software defined radio radiometer would have over a more traditional radiometer.  This was focused on looking at three main areas; cost, weight and size, and the value a SDR radiometer can add over traditional radiometers.

\subsection{Cost Benefits}
Software defined radios have become more commonplace in recent years and this has generated a number of Commercial Off The Shelf (COTS) solutions.  A COTS solution is often a lower cost solution due to the mass manufacturing that takes place.  This has driven the cost of many SDRs to under one thousand dollars while still having excellent performance characteristics.  The N200 SDR purchased for this research cost fifteen hundred dollars and the daughter-board cost one hundred and fifty dollars approximately.  Other software defined radios however have come out on the market since then.  Ettus for example has some that are below one thousand dollars and the author has also obtained the HackRF One SDR that now sells for three hundred dollars.  The main difference with the different software defined radios on the market is with both the resolution, or how many bits the ADC is, and the bandwidth they are able to handle.

\begin{table}[h!tb] \centering
\isucaption{Cost Analysis}
\label{cost_table}
% Use: \begin{tabular{|lcc|} to put table in a box
\begin{tabular}{lcc} \hline
\textbf{Device} & \textbf{Quantity} & \textbf{Cost} \\ \hline
\textbf{SDR Solution}& & \\ \hline
N200 SDR & 1 & \$1515 \\
LNA at \$60 ea. & 3 & \$180 \\
DBSRX2 Daughter-board & 1 & \$152 \\
GNURadio & 1 & \$0 \\ \hline
Total & & \$1847 \\ \hline
\textbf{ISU Radiometer} \\ \hline
LNA, FPGA, ADC, Microcontroller and power supplies & 1 & \$10,000\tablefootnote{Purchase price in 2005} \\ \hline
\textbf{Commercial Off the Shelf Unit}\\ \hline
Spectracyber 1420 MHz Hydrogen Line Spectrometer & 1 & \$2,650 \\ \hline

\end{tabular}
\end{table}

As see in table \ref{cost_table}, even the higher cost Ettus research equipment is a lower cost option than the custom built ISU radiometer purchased from University of Michigan and even a comparable off the shelf radiometer.  It should be noted that the radiometer from the University of Michigan is also a dual polarization radiometer so there are two RF front ends and two ADCs that feed into a FPGA board.  It would be quite easy to add dual polarization to the Ettus N200 SDR as it does support two daughter-boards.  This would increase the cost to \$2,179 for the additional LNAs and daughter-board.

The largest cost benefit is that key components that you find in a radiometer, the filters and square-law detector can now be all done in software instead of needed additional equipment.  The system is also much more frequency agile, which means it can work on a broader range of frequencies than most traditional radiometers with very little change in hardware and in some cases may require no change in hardware.  Some of this does depend on the SDR hardware however.  The Ettus N200 for example uses daughter-boards to provide the RF interface.  While these boards provide a high quality in the RF signal, it does come at a cost and are usually designed for certain bands of frequencies.  Other low cost SDRs however are also very wide range in the frequencies they will work in.  The HackRF for example works from 10 MHz to 6 GHz, but does so at the cost of lower resolution, less gain in its front end and a lower bandwidth that it can handle.

\subsection{Weight and component size benefits}

A typical radiometer has many components that are involved in the design of the radiometer.  This includes filters, LNAs and the power detection or square-law detector used.  These components add both weight, size and costs to the radiometer.  A software defined radio however digitizes the signal and we are able to replace the filters and square-law detector with their software equivalent.  While a software defined radio does add both the ADC and usually a FPGA to do the processing on the signal, advances in semiconductor technology has continued to shrink these components.  These components are also lighter than the filters often used in radiometers.

\begin{table}[h!tb] \centering
\isucaption{Weight Analysis}
\label{weight_table}
% Use: \begin{tabular{|lcc|} to put table in a box
\begin{tabular}{lc} \hline
\textbf{Device} & \textbf{Mass} \\ \hline
\textbf{SDR Solution} & \\ \hline
N200 SDR & 1.2 kg \\
LNA at .03 kg ea & .09 kg \\
DBSRX2 Daughter-board & .1 kg \\ \hline
Total & 1.39 kg \\ \hline
\textbf{ISU Radiometer} \\ \hline
LNA, FPGA, ADC, Microcontroller and power supplies & 22.7 kg \\ \hline
\textbf{Commercial Off the Shelf Unit}\\ \hline
Spectracyber 1420 MHz Hydrogen Line Spectrometer & 6 kg\tablefootnote{Estimated, no data available} \\ \hline

\end{tabular}
\end{table}

Size is another benefit as since semiconductor technology has continued to shrink components.  Again, since items like the filters and square-law detector are removed and done in software this helps to reduce the overall size.  

\subsection{Value added benefits}

A software defined radio radiometer adds additional value for two reasons.  One, it is able to work with both frequency and magnitude where most radiometers do not.  This allows for additional analysis on the signal and can help identify issues such as an interfering signal that was demonstrated in this thesis.  

Second, we are able to have an agile system that is able to adapt to changing conditions with very little or no change to hardware.  Different types of radiometers can be implemented such as a Dicke radiometer, dual polarization radiometer or a radiometer that can perform Stokes parameters.  In addition, since we have both frequency and power information we can create a system that is able to adapt to changing conditions such as dealing with an interfering signal.  

\section{Disadvantages of a SDR Radiometer}
Although we have outlined a number of advantages of using a COTS SDR Radiometer and how a SDR can add additional value to the radiometer system, there are some disadvantages to a SDR Radiometer.

\subsection{Power Consumption}
One of the largest drawbacks to a SDR radiometer can be in the power consumption of the SDR.  With the move to perform functions such as power detection and filtering we now require additional computational power to perform these tasks.  With those computational cycles additional power is now required.  The use of FPGAs and SoC however can help to minimize these power concerns as they are more efficient than using a full scale x86 based processor and on board computer system.  

Power and CPU requirements also increase as we add additional functionality such as filtering an offending signal.  While these additions may not require additional hardware, it can require additional processor or computational requirements.  This will cause additional strain on the processor and also in the memory requirements for the SDR as well.

\subsection{Bandwidth constraints}
While SDR technology has advanced, bandwidth is still a constraint that affects SDRs and in turn a SDR Radiometer.  Bandwidth plays a critical role in the radiometer's sensitivity as explained in this thesis, therefore the fact that many SDRs are limited in bandwidth does create a disadvantage.  In many cases this bottleneck takes place in both the transport and processing of large bandwidth systems.  This also relates to the power consumption disadvantage since larger bandwidth also means requiring additional computational cycles as well.  

In contrast, a square-law detector usually has a very large bandwidth, as much as one gigahertz, and is why we usually need to filter to the frequency band of interest.  

\section{Results Summary}

%----------------------------------------------------------
% End of Chapter 6.  Anything below this is extra information

%The data collected was then graphed using Excel.  The graph shown in \ref{adl5902_linear} shows that the ADL5902 has a linear output and matches the expected value based on the input.  

%\begin{figure}[h!tb] \centering
%\includegraphics[width=\textwidth]{Images/Linearsquarelaw}
%\isucaption{Graph showing the linearity of the ADL5902}
%\label{adl5902_linear}
%\end{figure}

%Once it was confirmed the ADL5902 does in fact have a linear response, we then looked at ways to work with the ADL5902.  The graph above was generated using Analog Devices program that talked to the Blackfin processor on the daughter-board.  However, the information for talking to the Blackfin is proprietary.  Therefore, the daughter-board was removed for future tests and instead we used the USB-6009 data acquisition board to read the raw analog voltages from the ADL5902.  This information was then stored later for additional analysis.  

