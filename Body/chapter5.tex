% Chapter 5 from the standard thesis template
%   with a full page figure and a sideways table.

%Chapter 5 discusses the Experimental setup and experiments used and how I evaluated the performance of the radiometer.

\chapter{EVALUATION SETUP AND EXPERIMENTAL DESIGN}\label{ch:exp_design}
The experiments outlined in this chapter are used to demonstrate and verify that a software defined radio (SDR) base radiometer can perform on par with a traditional radiometer.  In addition, these experiments are designed to demonstrate that a SDR based radiometer can provide additional functionality not typically found in traditional radiometers. 

Four experiments are described.  Experiment one verifies our SDR-based radiometer by comparing its operation to a square-law detector, a device typically used within a traditional radiometer.  Experiment two evaluates the sensitivity and stability of our SDR-based radiometer.  Experiment three will evaluate our SDR-based radiometer's ability to mitigate an interfering signal in comparison to the behavior of a traditional radiometer in the presence of an interfering signal.  Experiment four's purpose is to further examine the impact of our frequency based notching approach for RFI mitigation on radiometer sensitivity.

\section{Experiment I - Software Defined Radiometer Verification and Calibration}\label{Exp1}
%Experiment 1
%Why is this experiment being run?
%What type of data will you collect?
%What is the experimental setup?

This experiment is designed to verify our SDR-based radiometer functions as expected.  Its behavior is compared against a square-law detector, which is commonly used in traditional radiometers.  To verify the results of the information that the software defined radio is obtaining a square-law detector is used to measure the power of the incoming signal in parallel to the SDR-based radiometer.  This signal is split using a power divider so that the information will be the same to both devices.  This allows us to verify the software defined radio with a proven system.  

\subsection{Experimental setup} \label{exp1_setup}

Figure \ref{Exp1_Block} shows a block diagram of the experimental setup.  A matched load is used to simulate our source signal.  This matched load is then submerged in temperature baths.  These baths use Liquid Nitrogen (LN2) which is known to boil at 77 Kelvin and an ice water bath which is known to be at 273.15 Kelvin.  The temperature of these could easily be monitored and maintained.  The load was submersed in each bath for a minimum of 2 minutes to allow for it to reach the same temperature as the bath.  The physical temperature of this matched load is then the noise temperature the radiometer sees and can be used to calibrate the radiometer.  

{\begin{figure}[h!tb] \centering
\includegraphics[width=\textwidth]{Images/Exp_1_Setup.pdf}
\isucaption{Block diagram of Experiment 1 setup.  Source: Matthew E. Nelson}
\label{Exp1_Block}
\end{figure}
}

The radiometer RF front end provides the amplification needed for our experiments.  Figure \ref{ISURF} shows an image of the RF front end used with the LNAs and band-pass filters marked.  After the signal has been amplified we divide that signal between the ADL5902 square-law detector and the N200 Software defined radio.  The N200 is then connected to a personal computer running XUbuntu Linux and GNURadio.

{\begin{figure}[h!tb] \centering
\includegraphics[width=\textwidth]{Images/ISU_RF_noted.jpg}
\isucaption{The radiometer RF front end with LNAs and band-pass filters used in the experiments.  Source: Matthew E. Nelson}
\label{ISURF}
\end{figure}
}

The following hardware was used and has also been marked on Figure \ref{Exp1_Block}:

\begin{enumerate}
\item N200 Software Defined Radio with DBSRX2 Daughter-board
\item ADL5902 Square-law detector
\item National Instruments USB-6009 Data Acquisition Unit
\item ZN2PD-20-S+ Power Divider
\item 50-ohm matched load
\item Rigol DP832 Power Supply
\item Radiometer RF Front End
\begin{enumerate}
\item 4 x Integrated Microwave Bandpass filters (1400 - 1425 MHz)
\item 2 x Miteq AMF-3F-01400147-30-10P LNA
\item 1 x Miteq AMF-2F-01400147-04-10P LNA
\end{enumerate}
\end{enumerate}

%The radiometer RF front end, as shown in figure \ref{ISURF} was used as it has the necessary Low Noise Amplifiers (LNAs) to provide the needed amplification.  The only item powered for these experiments was the LNAs, all other components are passive or were deactivated.  The radiometer RF front end uses the following hardware:

\subsection{Data Collection}\label{exp1_data}

Two sets of data are produced with this experiment.  First, data is generated from the software defined radio using GNURadio.  Second, data is generated from a data acquisitions device that is attached to the square-law detector.  Data from each is stored to a local computer running the appropriate software.

\emph{Software Defined Radio Data.}  The data from the software defined radio is stored in files generated from GNURadio.  GNURadio uses a sink block to output the data to either a screen, socket connection such as TCP/IP or a file.  As shown in Figure \ref{filesink}, a file sink block is used to output the data to a file.  The flow of data to this sink is controlled by a valve block.  This allows for the user to turn on and off recording of the data.

{\begin{figure}[h!tb] \centering
\includegraphics[width=5cm]{Images/TPR_Filesink.png}
\isucaption{The File Sink block used in GNURadio.  Source:  GNURadio}
\label{filesink}
\end{figure}
}

There are two types of files that the SDR generates.  The first type of file is the I (in-phase) and Q (quadrature phase) data points.  This file is stored in little-indian format as complex values.  Due to the sampling rate, it is not uncommon for this file to grow quite large, usually several gigabytes of data for a 10-15 minute run.  However, this file can then be feed back through GNURadio later to be played back if needed, and it contains the information needed to completely recreate the signal.

The second file type is the total power values generated from the total power block in GNURadio.  A diagram of this block can be found in Appendix \ref{appendix1} (Figure \ref{TPR_GRC}), and its source code can be found in Appendix \ref{appendix1}.  This file also uses a little-endian format, however this file only has real values.  This file is also much smaller than the file that contains the I and Q data points.  A typical file size is 50 - 100 kB for a 10 to 15 minute run.  

\emph{Square-law detector data.}  A square-law detector outputs information as an analog voltage that is linearly proportional to the RF power measured.  The Analog Devices ADL5902 is a single Integrated Circuit (IC) that contains a square-law detector and necessary amplification for the output signal and is shown in Figure \ref{square_law} .  This device operates from 50 MHz to 9 GHz and can detect power as low as -60 dBm.  The output voltage from the square-law is amplified to a range from zero to five volts with a calibrated output of 53 mV/dB.  

{\begin{figure}[h!tb] \centering
\includegraphics[width=7cm]{Images/adl5902.jpg}
\isucaption{The ADL50902 IC on a demonstration board.}
\label{square_law}
\end{figure}
}

To capture the voltage output from the ADL5902, a data acquisition unit (DAQ) is used.  The National Instruments USB-6009 DAQ unit was selected as it met the requirements for an easy to use yet high enough resolution to obtain accurate information.  The USB-6009 unit has 8 analog inputs that can sample at 48 KSPS with a resolution of 14-bits.  

To use the USB-6009 a fairly simple Lab View program was created to obtain, display and store the data from the ADL5902.  This program retrieved the information from the USB-6009 and stored the data in both Labview's binary format and in a more human friendly ASCII format.  The USB-6009 then connects to a host computer through the USB interface.  A GUI program, shown in Figure \ref{labviewgui}, is then used to control and display the data.  This made obtaining the data and using the device straightforward.

{\begin{figure}[h!tb] \centering
\includegraphics[width=\textwidth]{Images/labviewGUI.png}
\isucaption{A screenshot of the Labview GUI interface.  Source:  Labview}
\label{labviewgui}
\end{figure}
}

For rapid development, the National Instruments DAQ assistant was used to quickly configure and setup the USB-6009.  Labview also includes blocks that allows us to easily record the data to a file and to use a low pass filter.  These blocks made up most of the program and resulted in a program that was quickly made.  Figure \ref{labviewblock} shows the blocks used and the wiring of the blocks.

{\begin{figure}[h!tb] \centering
\includegraphics[width=\textwidth]{Images/labview-diagram.png}
\isucaption{A screenshot of the Labview block diagram.  Source:  Labview}
\label{labviewblock}
\end{figure}
}

\section{Experiment II - Verification of sensitivity and stability}\label{Exp2}

In this experiment we will verify the sensitivity and stability of a software defined radiometer.  This experiment will verify that a software defined radiometer is able to meet the expected sensitivity and stability within an acceptable range.    

\subsection{Experimental setup} \label{exp2_setup}
The experimental setup used for experiment two is the same experiment as outlined in section \ref{exp1_setup}.  

\subsection{Data Collection}
The data collected for this experiment was the total power measurements made from the software defined radiometer.  These measurements uses the same method as outlined in section \ref{exp1_data}.

\section{Experiment III - Interfering Signal Mitigation}\label{Exp3}

In this experiment we generate an interfering signal and then mitigate the signal using a software defined filter.  A square-law detector is hooked up in parallel to measure the same signal but is not provided with a mitigation mechanism.  We then compare the two signals to verify that the SDR-based radiometer can mitigate the interfering signal, while continuing to still make useful total power measurements.

This experiment was designed to determine whether or not a SDR-based radiometer can cope with an interfering signal.  This test injects a known signal at 1.406 GHz to interfere with the normal operation of the radiometer.  The amplitude of this signal is then incremented and decremented at various times in the test.  This was done to reflect a possible real world scenario and to make it easy to identify the interfering signal with the square-law detector which only measures power.  

In order to mitigate the offending signal, a filter is designed to remove the offending signal.  The design of the filter used a program that is part of the GNURadio software package, called the GNU Radio Filter Design tool.  Figure \ref{GRC_Filter_DSN} shows a screen shot of this tool when designing the band-reject filter for this application.  This tool generates filter values (also called taps) that GNURadio will use for defining the filter.  The GUI program shown in figure \ref{GRC_Filter_DSN} allows us to interactively create a filter.  Because this tool is part of the GNURadio package, it also includes a command line interface to the program.  This allows us to call the program from within GNURadio to integrate this functionality into our SDR-based radiometer.  

\begin{figure}[h!tb] \centering
\includegraphics[width=\textwidth]{Images/GNURadio_Filter_dsn.png}
\isucaption{Image of the GNU Radio Filter Design tool}
\label{GRC_Filter_DSN}
\end{figure}  

\subsection{Experimental setup} \label{exp3_setup}

The setup for this experiment is similar to the setup outline in section \ref{Exp1}.  In addition a second software defined radio was added to inject an offending signal into the RF signal chain.  This software defined radio was configured to operate as a signal generator to create the offending signal.  

Our signal generator is the HackRF One (or just HackRF), shown in figure \ref{HackRF}.  This SDR is cheaper, and has lower specifications than the Ettus Research N200 used as the SDR-based radiometer.  However, it suits our needs for the purpose of a signal generator.

\begin{figure}[h!tb] \centering

\includegraphics[width=\textwidth]{Images/Hack_RF.jpg}
\isucaption{Image of the HackRF used to generate the offending signal}
\label{HackRF}
\end{figure} 

The HackRF generates a sinusoidal wave at a fixed frequency, and the signal amplitude changed at different times during the experiment.  This is controlled from a program called \emph{osmocom\_siggen}.  Osmocom was originally developed to communicate with OsmocomSDR hardware.  However, it has been expanded to include the HackRF and Ettus Research hardware.  The \emph{osmocom\_siggen} program provides a GUI to set the frequency, amplitude, type of the signal generated.  

\subsection{Data Collection}

The data collected for this experiment includes both the total power measurements from the software defined radiometer and the square-law data.  Section \ref{exp1_data} explains in detail the setup and configuration of the equipment used to collect this data.

\section{Experiment IV - Performance impact of interfering signal mitigation}\label{Exp4}

In this experiment we examine the impact of filtering an interfering signal on the sensitivity of the SDR-based radiometer and how reducing our overall bandwidth affects our total power received to the radiometer.

\subsection{Experimental setup} \label{exp4_setup}

In this experiment we setup our experiment as outlined in section \ref{exp3_setup}.  The rest of the data collected is done by the analysis of the data collected.

\subsection{Data Collection}

The data collected for this experiment is the total power reading measurements obtained from the software defined radio.  This data is identical to the method used in section \ref{exp1_data}.

%----------------------------------------------------------
% End of Chapter 5.  Anything below this is extra information

%To verify the results of the information that the software defined radio is obtaining a square-law detector is used to measure the power of the incoming signal in parallel to the software defined radio.  This signal is split using a power divider so that the information will be the same to both devices with the except for the 3 dB plus insertion loss the power divider adds.  This allows us to verify the software defined radio with a proven system. 

%To test the ADL5902 a signal generator was used that had a controlled output.  The specific signal generator that was used was an older model that allowed us to change the output in 10 dBm increments.  The signal generator was also configured for 1.4 GHz as that is the frequency the ISU RF front end is configured to listen at.  Ideally a noise generator would have been desirable, however a noise generator with adjustable power output could not be located on campus.  

%The ADL5902 is available from Analog Devices in an evaluation board.  This board pairs the ADL5902 with a AD7466 12-bit analog to an analog to digital converter.  This board can then be mated with Analog Devices BlackFin processor which acts as a USB gateway for the AD7466 data.  A test program written in LabView is also provided as well.

%The test program provided by Analog Devices allows us to query the ADL5902 and record the raw ADC value.  This program also allows us to enter in the frequency used during testing, the temperature during the test, and allowed for calibration of the ADL5902.  All of the data is then stored into an Excel spreadsheet which can be accessed later.

%For this test we used the signal generator set to 1.4 GHz and started at $-60$ dBm for the output signal.  This was selected as this is the lowest the square law detector can detect.  The output power was then incremented on the signal generator in 10 dBm steps.  There was no change to any other parameter.  This was done up to 0 dBm.  Any higher and there was risk of damage to the ADL5902.  This test was then repeated several times and was done with the signal generator stepping up from $-60$ dBm to 0 dBm and from 0 dBm to $-60$ dBm.  

%The square-law detector that we obtained is the Analog Devices ADL5902.  The ADL5902 is a true rms responding power detector that has a square law detector, a variable gain amplifier, and an output driver. It also has a temperature sensor and will compensate for temperature variations.  The output driver allows for the small signal from the square law detector to be amplified to a level that most analog to digital devices can detect.  It should be noted that this is not an amplifier for the incoming RF signal, just to amplify the small signal from the diode.  This driver however does have low noise and has a noise output of approximately $25nV/ \sqrt{Hz}$ at 100 kHz.  The ADL5902 operates from 50 MHz to 9 GHz and in most cases can detect down to $-60$ dBm.  This works well in our application since the radiometer operates at 1.4 GHz and after the amplification stage we usually see between $-40$ to $-30$ dBm of power.  

%{\begin{figure}[h!tb] \centering
%\includegraphics[width=\textwidth]{Images/adl5902.jpg}
%\isucaption{An image of the ADL5902 square-law detector used in these experiments}
%\label{ADL5902}
%\end{figure}
%}

%The specifications of the ADL5902 can be seen in Table~\ref{ADL5902_data} shown below.

%\begin{table}[h!tb] \centering
%\isucaption{ADL5902 Specifications}
%\label{ADL5902_data}
%\begin{tabular}{lcc} \hline
%\textbf{Parameter @ 900 MHz} & \textbf{Value} & \textbf{Units} \\ \hline
%Frequency Range & 50 to 9000 & MHz \\
%Dynamic Range & 61 & dB \\
%Minimum Input Level, $\pm 1.0$ dB & 60 & dB \\ 
%Maximum Input Level, $\pm 1.0$ dB & 1 & db \\
%Logarithmic Slope & 53.7 & mV/dB \\ 
%Output Voltage Range & 0.03 to 4.8 & V \\ \hline
%\end{tabular}
%\end{table}

%The ADL5902 outputs an analog signal that falls between 0.03 volts and 4.8 volts.  It outputs a change of 53.7 millivolt per dB detected by the ADL5902. 