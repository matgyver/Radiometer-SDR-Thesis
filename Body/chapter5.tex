% Chapter 5 from the standard thesis template
%   with a full page figure and a sideways table.
\chapter{PERFORMANCE AND EVALUATION}

Radiometers measure power and it is this information that we can use to determine information about a certain target.  This power however is often expressed in terms of an equivalent temperature and if we are looking at an object, this is the brightness temperature of that object.  The primary function of the radiometer is to measure the power that is seen at the radiometer's antenna.  Ideally this is the brightness temperature of an object of interest.  In our case, this is usually looking at the ground or soil sample.  Thus the goal of the radiometer is to measure this antenna temperature with sufficient resolution and accuracy that a correlation can be made between the antenna temperature and the object's temperature that we are studying.

\section{Performance of a radiometer}
The performance of a radiometer can be measured by looking at both the sensitivity and the accuracy of the radiometer.  In addition, we need to be concerned with the stability of the radiometer as this affects the accuracy of the radiometer.


\subsection{Sensitivity}
Sensitivity of the radiometer relates to the amount of power that the radio selects from the antenna.  This selection is then dependent on the bandwidth that the radiometer is able to listen to.  The radiometer however detects not only the signal of interest but also receives a noise signal as well.  This noise is added to the signal and can not be separated from the signal.  Because this noise is added to the signal, we must be able to determine a change in the signal while the noise signal is also present.  

Power from the radiometer can be expressed in equation \ref{eq:final_power} from chapter 2, where we take into consideration bandwidth, gain and the input from the antenna plus the noise added from the antenna.  

Sensitivity relates to being able to distinguish one signal from the other.  In other words, we must be able to distinguish, or detect, our wanted signal from noise that is present in the signal.  To demonstrate this, consider a system that has a system noise temperature of 700K and an antenna temperature of 300K.  This gives us a total noise temperature of 1000K.  Therefore, if we wish to detect a change of 1K, we need to be able to detect between 1000K and 1001K.

Since our signal is really random fluctuations about a mean power input to our system, we can reduce the fluctuations by averaging or integrating the signal.  This results in equation \ref{eq:rad_sensitivity} which is derived in chapter 2.

\begin{equation} \label{eq:rad_sensitivity}
\Delta T=\frac{T_{A}+T_{N}}{\sqrt{\beta * \tau}}
\end{equation}


This equation gives us the radiometer sensitivity based on the input, which is both the noise and input signal with consideration to the bandwidth and integration time, or averaging, that is done to the signal.  

\subsection{Accuracy and Stability}
Stability and accuracy are additional problems that need to be considered when looking at the radiometer system.  To begin we can once again look at the power received equation that we discussed in Chapter 2 and is equation \ref{eq:final_power}.

As we look at this equation, we can see that if k, B, G, and $T_{N}$ are constant, then stability can be assured.  k is a known constant and we can also assume that our bandwidth, B, will also remain constant, or at least while we are taking our measurements.  Gain and the noise temperature however can vary.  

Gain is usually our largest factor that can change on a radiometer and even with a software defined radio this is still a large source of variation.  This is due to the analog nature of the amplifiers that affect a large portion of the gain in our system.  Various things can affect our gain, but the two largest factors is the physical temperature of the amplifier and the voltage that feeds the amplifier.  Voltage can be controlled to a degree.  High accuracy voltage regulators can help control fluctuations in voltages that can in turn affect the gain.  A factor however that is harder to control is temperature.  It is because of this that the current ISU radiometer has gone to great strides to control the temperature of the amplifiers to maintain a constant temperature.

\section{Required Performance Requirements}

To help us quantify the required performance of the radiometer, we looked to what the existing ISU radiometer had in terms of performance.  These performance numbers were in part determined by Dr. Brian Hornbuckle but also derived from existing radiometers.  As stated earlier, although a radiometer measures power, we often convert this to an equivalent brightness temperature.  Specifically, we are looking at the brightness temperature of the antenna added to the brightness temperature of the object of interest.  We also have to be able to detect a minimum amount of power.  Since changes in noise can be small, the better the sensitivity of the radiometer, the better we can detect these small changes.  

The requirements given are outlined in the table below.

\begin{table}[h!tb] \centering
\isucaption{Required Radiometer performance}
\label{rad_performance}
% Use: \begin{tabular{|lcc|} to put table in a box
\begin{tabular}{lcc} \hline
\textbf{Parameter} & \textbf{Value} & \textbf{Units} \\ \hline
Minimum bandwidth & 20 & MHz \\
Operational frequency & 1400 - 1425 & MHz \\
$NE\Delta T$ & 1 & Kelvin \\ \hline
\end{tabular}
\end{table}

\section{Square-law Detector Performance}
The Square-law detector was added to our system in order to give us another reference point and to help verify the power output that the software defined radio.  The performance of our square-law detector is based on two items.  The first is the actual square-law detector itself.  The sensitivity of this device accounts for most of the performance factor of the system.  In our system the output of this square-law detector is then feed directly into an analog to digital converter.  Therefore the performance of this A/D converter needs to be accounted for as well [\cite{Terlep}].  

For our square-law detector, it has a noise output of $25nV/ \sqrt{Hz}$ at 100 kHz and will detect a signal as low as $-60$ dBm.  This works will with our needs since the RF front end brings the noise floor to approximately $-30$ dBm.


\section{Software Defined Radio Performance}
Performance of the software defined radio is governed by the system that takes in the RF signal and then digitizes the signal.  Once the signal is in digital form, we no longer are concerned about loss of performance due to additional noise that may get added to the system.  For this reason, we attempt to digitize the signal as soon as possible.

For the N200 this is done by the DBSRX2 daughter-board that plugs into the N200 base system.  While this board does play a part in the overall system performance, because this sits later in the chain however the impact to the performance is low as shown in equation \ref{noise_factor}.  However, this module does need to be in the frequency range of the radiometer, or in our case 1.4 GHz.  This board meets that criteria.  Ideally, we still want the noise figure on this board as low as possible, even though the impact is low.  The DBSRX2 does meet this having a noise figure of approximately 5 dBm.
