% Chapter 5 from the standard thesis template
%   with a full page figure and a sideways table.
\chapter{PERFORMANCE AND EVALUATION}

Radiometers measure power and it is this information that we can use to determine information about a certain target.  This power however is often expressed in terms of an equivalent temperature and if we are looking at an object, this is the brightness temperature of that object.  The primary function of the radiometer is to measure the power that is seen at the radiometer's antenna.  Ideally this is the brightness temperature of an object of interest.  In our case, this is usually looking at the ground or soil sample.  Thus the goal of the radiometer is to measure this antenna temperature with sufficient resolution and accuracy that a correlation can be made between the antenna temperature and the object's temperature that we are studying.

\section{Performance of a radiometer}
The performance of a radiometer can be measured by looking at both the sensitivity and the accuracy of the radiometer.  In addition, we need to be concerned with the stability of the radiometer as this affects the accuracy of the radiometer.

\subsection{Sensitivity}
Look at section 3.2 in book

\subsection{Accuracy and Stability}
Look at section 3.3 in book


\section{Required Performance Requirements}

To help us quantify the required performance of the radiometer, we looked to what the existing ISU radiometer had in terms of performance.  These performance numbers were in part determined by Dr. Brian Hornbuckle but also derived from existing radiometers.  As stated earlier, although a radiometer measures power, we often convert this to an equivalent brightness temperature.  Specifically, we are looking at the brightness temperature of the antenna added to the brightness temperature of the object of interest

%Or graphically as seen in Figure~\ref{mgraph2}
%it is certain that my hypothesis is true.


\section{Square-law Detector Performance}
The Square-law detector was added to our system in order to give us another reference point and to help verify the power output that the software defined radio.  The performance of our square-law detector is based on two items.  The first is the actual square-law detector itself.  The sensitivity of this device accounts for most of the performance factor of the system.  In our system the output of this square-law detector is then feed directly into an analog to digital converter.  Therefore the performance of this A/D converter needs to be accounted for as well.  


\section{Software Defined Radio Performance}

A summary of the performance of the SDR and this can conclude the results.


