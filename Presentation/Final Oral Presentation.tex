%%%%%%%%%%%%%%%%%%%%%%%%%%%%%%%%%%%%%%%%%
% Thesis Presentation
% Software Defined Total Power Radiometer
% Version 0.1
% Matthew E. Nelson
% Iowa State University
% 
% License:
% CC BY-NC-SA 3.0 (http://creativecommons.org/licenses/by-nc-sa/3.0/)
%
%%%%%%%%%%%%%%%%%%%%%%%%%%%%%%%%%%%%%%%%%

%----------------------------------------------------------------------------------------
%	PACKAGES AND THEMES
%----------------------------------------------------------------------------------------

\documentclass{beamer}

\mode<presentation> {

%Beamer Themes, uncomment the theme you wish to use

%\usetheme{default}
%\usetheme{AnnArbor}
%\usetheme{Antibes}
%\usetheme{Bergen}
%\usetheme{Berkeley}
%\usetheme{Berlin}
%\usetheme{Boadilla}
\usetheme{CambridgeUS}
%\usetheme{Copenhagen}
%\usetheme{Darmstadt}
%\usetheme{Dresden}
%\usetheme{Frankfurt}
%\usetheme{Goettingen}
%\usetheme{Hannover}
%\usetheme{Ilmenau}
%\usetheme{JuanLesPins}
%\usetheme{Luebeck}
%\usetheme{Madrid}
%\usetheme{Malmoe}
%\usetheme{Marburg}
%\usetheme{Montpellier}
%\usetheme{PaloAlto}
%\usetheme{Pittsburgh}
%\usetheme{Rochester}
%\usetheme{Singapore}
%\usetheme{Szeged}
%\usetheme{Warsaw}

%Beamer color themes, select one, or use none to use the themes default color scheme

%\usecolortheme{albatross}
%\usecolortheme{beaver}
%\usecolortheme{beetle}
%\usecolortheme{crane}
%\usecolortheme{dolphin}
%\usecolortheme{dove}
%\usecolortheme{fly}
%\usecolortheme{lily}
%\usecolortheme{orchid}
\usecolortheme{rose}
%\usecolortheme{seagull}
%\usecolortheme{seahorse}
%\usecolortheme{whale}
%\usecolortheme{wolverine}

% To remove the footer line in all slides uncomment this line
%\setbeamertemplate{footline} 
% To replace the footer line in all slides with a simple slide count uncomment this line
%\setbeamertemplate{footline}[page number] 
% To remove the navigation symbols from the bottom of all slides uncomment this line
%\setbeamertemplate{navigation symbols}{} 
}

\usepackage{graphicx} 
% Allows the use of \toprule, \midrule and \bottomrule in tables
\usepackage{booktabs} 

%----------------------------------------------------------------------------------------
%	TITLE PAGE
%----------------------------------------------------------------------------------------

% The short title appears at the bottom of every slide, the full title is only on the title page

\title[SDR Radiometer]{Implentation of a Total Power Radiometer in Software Defined Radios} 

\author{Matthew E. Nelson} 
\institute[ISU] 
{
% Your institution for the title page
Iowa State University \\ 
\medskip
% email address or contact
\textit{mnelson@iastate.edu} % Your email address
}
\date{\today} 

\begin{document}

\begin{frame}
% Print the title page as the first slide
\titlepage 
\end{frame}

\begin{frame}[allowframebreaks]
% Table of contents slide, comment this block out to remove it
\frametitle{Overview} 
% Anything with a \section or \subsection will get populated here.  allowframebreaks allow for this to go to more than one slide for large presentations
\tableofcontents 
\end{frame}

%----------------------------------------------------------------------------------------
%	PRESENTATION SLIDES
%----------------------------------------------------------------------------------------

%------------------------------------------------
\section{Introduction} 

%Each begin frame starts a new slide
\begin{frame}
\frametitle{Introduction}
The goal of this thesis and presentation is to explore other methods that could be used for a remote sensing radiometer and specifically the implementation of a software defined radio as a total power radiometer.  The end goal is to develop a radiometer that is more flexible than most radiometers and still maintain the accuracy and stability of a traditional radiometers if not exceed these specifications.  
\end{frame}

\begin{frame}
A secondary goal was to use off the shelf components and components that are generally more accessible.  This would allow radiometers to be more accessible to a wider scope of researchers in this field. 
\end{frame}

\begin{frame}
And finally a tertiary goal was to ensure that the system as a whole is fairly easy to use.  This ties to our secondary goal of making radiometers more accessible to a wider range of researchers and research topics.
\end{frame}

\begin{frame}
As stated in the background, ISU currently owns a radiometer that is used for the remote sensing of soil moisture.  In many ways, the current ISU radiometer is like a SDR.  The RF signal is digitized and then processed by a FPGA to give the total power reading needed for radiometry.  However, due to the under-sampling of the signal some information is lost.  The ISU radiometer assumes that incoming signal is free of any interference and the power recorded is from the target of interest.  However, in recent years the ISU radiometer has not been performing as expected, and without additional information it is almost impossible to diagnose these issues out in the field.  The ISU radiometer is also not frequency agile, it is fixed at 1.4 GHz and the bandwidth is also fixed at 20 MHz.  All of this means that if an interfering signal is present the current radiometer would not be able to determine this and even if it could, it has no means to try to avoid the signal.  

This thesis looks to explore the following questions: (1) Can we use a SDR along with GNURadio to recreate a radiometer in software?  (2) If so, what performance can we get from the system?  (3) What benefits do we gain (if any) from using a SDR from a more traditional radiometer?  The results of this research and experimentation are the subject of this thesis.
\end{frame}

\subsection{Software Defined Radio} 

\subsection{GNURadio Software}

\subsection{Current ISU Radiometer}

\subsection{Related Works}

\section{General Theory of Operation}

\subsection{Software Defined Radios}

\subsection{GNURadio Software}

\subsection{RF Front End}

\section{GNURadio and N200 TPR}

\subsection{Hardware}

\subsubsection{Requirements}

\subsubsection{Noise Temperature Considerations}

\subsubsection{Existing ISU Radiometer RF Front End}

\subsection{Software}

\subsubsection{Requirements}

\subsubsection{Obtaining Signal Information and GUI Controls}

\subsubsection{Implantation of a TPR in Software}

\subsubsection{Data Storage and Display}

\section{Testing and Verification}

\subsection{Square-law Detector}

\subsection{SDR Tests}

\subsection{E E 518 Lab Tests}

\subsection{LN2 Tests}

\section{Performance and Evaluation}

\subsection{Sensitivity, Accuracy and Stability}

\subsection{Required Performance}

\subsection{Square-law Detector Performance}

\subsection{SDR Performance}

\section{Conclusion and Future Work}

\begin{frame}
\frametitle{Paragraphs of Text}
Sed iaculis dapibus gravida. Morbi sed tortor erat, nec interdum arcu. Sed id lorem lectus. Quisque viverra augue id sem ornare non aliquam nibh tristique. Aenean in ligula nisl. Nulla sed tellus ipsum. Donec vestibulum ligula non lorem vulputate fermentum accumsan neque mollis.\\~\\

Sed diam enim, sagittis nec condimentum sit amet, ullamcorper sit amet libero. Aliquam vel dui orci, a porta odio. Nullam id suscipit ipsum. Aenean lobortis commodo sem, ut commodo leo gravida vitae. Pellentesque vehicula ante iaculis arcu pretium rutrum eget sit amet purus. Integer ornare nulla quis neque ultrices lobortis. Vestibulum ultrices tincidunt libero, quis commodo erat ullamcorper id.
\end{frame}

%------------------------------------------------

\begin{frame}
\frametitle{Bullet Points}
\begin{itemize}
\item Lorem ipsum dolor sit amet, consectetur adipiscing elit
\item Aliquam blandit faucibus nisi, sit amet dapibus enim tempus eu
\item Nulla commodo, erat quis gravida posuere, elit lacus lobortis est, quis porttitor odio mauris at libero
\item Nam cursus est eget velit posuere pellentesque
\item Vestibulum faucibus velit a augue condimentum quis convallis nulla gravida
\end{itemize}
\end{frame}

%------------------------------------------------

\begin{frame}
\frametitle{Blocks of Highlighted Text}
\begin{block}{Block 1}
Lorem ipsum dolor sit amet, consectetur adipiscing elit. Integer lectus nisl, ultricies in feugiat rutrum, porttitor sit amet augue. Aliquam ut tortor mauris. Sed volutpat ante purus, quis accumsan dolor.
\end{block}

\begin{block}{Block 2}
Pellentesque sed tellus purus. Class aptent taciti sociosqu ad litora torquent per conubia nostra, per inceptos himenaeos. Vestibulum quis magna at risus dictum tempor eu vitae velit.
\end{block}

\begin{block}{Block 3}
Suspendisse tincidunt sagittis gravida. Curabitur condimentum, enim sed venenatis rutrum, ipsum neque consectetur orci, sed blandit justo nisi ac lacus.
\end{block}
\end{frame}

%------------------------------------------------

\begin{frame}
\frametitle{Multiple Columns}
\begin{columns}[c] % The "c" option specifies centered vertical alignment while the "t" option is used for top vertical alignment

\column{.45\textwidth} % Left column and width
\textbf{Heading}
\begin{enumerate}
\item Statement
\item Explanation
\item Example
\end{enumerate}

\column{.5\textwidth} % Right column and width
Lorem ipsum dolor sit amet, consectetur adipiscing elit. Integer lectus nisl, ultricies in feugiat rutrum, porttitor sit amet augue. Aliquam ut tortor mauris. Sed volutpat ante purus, quis accumsan dolor.

\end{columns}
\end{frame}

%------------------------------------------------
\section{Second Section}
%------------------------------------------------

\begin{frame}
\frametitle{Table}
\begin{table}
\begin{tabular}{l l l}
\toprule
\textbf{Treatments} & \textbf{Response 1} & \textbf{Response 2}\\
\midrule
Treatment 1 & 0.0003262 & 0.562 \\
Treatment 2 & 0.0015681 & 0.910 \\
Treatment 3 & 0.0009271 & 0.296 \\
\bottomrule
\end{tabular}
\caption{Table caption}
\end{table}
\end{frame}

%------------------------------------------------

\begin{frame}
\frametitle{Theorem}
\begin{theorem}[Mass--energy equivalence]
$E = mc^2$
\end{theorem}
\end{frame}

%------------------------------------------------

\begin{frame}[fragile] % Need to use the fragile option when verbatim is used in the slide
\frametitle{Verbatim}
\begin{example}[Theorem Slide Code]
\begin{verbatim}
\begin{frame}
\frametitle{Theorem}
\begin{theorem}[Mass--energy equivalence]
$E = mc^2$
\end{theorem}
\end{frame}\end{verbatim}
\end{example}
\end{frame}

%------------------------------------------------

\begin{frame}
\frametitle{Figure}
Uncomment the code on this slide to include your own image from the same directory as the template .TeX file.
%\begin{figure}
%\includegraphics[width=0.8\linewidth]{test}
%\end{figure}
\end{frame}

%------------------------------------------------

\begin{frame}[fragile] % Need to use the fragile option when verbatim is used in the slide
\frametitle{Citation}
An example of the \verb|\cite| command to cite within the presentation:\\~

This statement requires citation \cite{p1}.
\end{frame}

%------------------------------------------------

\begin{frame}
\frametitle{References}
\footnotesize{
\begin{thebibliography}{99} % Beamer does not support BibTeX so references must be inserted manually as below
\bibitem[Smith, 2012]{p1} John Smith (2012)
\newblock Title of the publication
\newblock \emph{Journal Name} 12(3), 45 -- 678.
\end{thebibliography}
}
\end{frame}

%------------------------------------------------

\begin{frame}
\Huge{\centerline{The End}}
\end{frame}

%----------------------------------------------------------------------------------------

\end{document} 